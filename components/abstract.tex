% !TEX root = ../main.tex
% The abstract.
% Included by MAIN.TEX
\clearemptydoublepage
\phantomsection
\addcontentsline{toc}{chapter}{Abstract}

\vspace*{2cm}
\begin{center}
{\Large \textbf{Abstract}}
\end{center}
\vspace{1cm}

\chapter*{Abstract}
\label{chapter:Abstract}

\ac{rl} has made significant progress in recent years, but it still faces challenges in achieving sample efficiency for tasks that involve high-dimensional continuous state and action spaces. 
This thesis proposes a new \ac{rl} paradigm called \ac{avc} that leverages planning with a learned model of the environment and expert demonstrations to improve performance on such challenging tasks. 
\ac{avc} is a general-purpose algorithm applicable to both deterministic \ac{mdp}s and deterministic \ac{pomdp}s. To the author's knowledge, it is the first algorithm to incorporate expert knowledge 
into inference-time planning on continuous action spaces without a given model of the environment.

The efficiency of our planning algorithm enables fast inference, which is necessary for real-world applications where steps must be computed in real-time. In this thesis, we investigate 
\ac{avc} on a setup with a single observation per trajectory, which challenges the learned model of the \ac{mdp}. This setup is particularly interesting in real-world scenarios where the 
robot is not given enough time to compute the input signal after each step of the trajectory. We also conduct experiments on an \ac{mdp} with observations after each step, where we incorporate 
inference time constraints. We provide a thorough theoretical analysis of the paradigm and establish a lower improvement bound that includes expert demonstrations.

We compare \ac{avc}'s performance to state-of-the-art \ac{rl} algorithms and a state-of-the-art imitation learning algorithm. Our experimental results show that \ac{avc} outperforms the selected 
baselines in both the \ac{sopomdp}s and \ac{mdp}s, achieving superior performance with improved sample efficiency. We also investigate \ac{avc} in tasks without expert demonstrations and 
find superior stability and sample efficiency, showcasing the generality of the approach.

Overall, the results demonstrate the efficacy of the \ac{avc} paradigm in incorporating expert knowledge and planning to achieve efficient learning and improved stability in challenging \ac{rl} tasks.