% !TEX root = ../main.tex

\chapter{Background}
\label{chapter:Background}
In this thesis machine learning will be defined as the guided search for candidate functions $f^*: X \to Y$ to maximize the expectation of an objective function 
${g: X \times Y \to \mathbb{R}}$ using a signal function $g^*$, which is available during training time:
\begin{equation}
    \label{general_learning_paradigm}
    \begin{aligned}
        f^* &= \arg\max_{f} \mathbb{E}_{x \sim D}[g(x,f(x))] \\
        f^* &\approx f' = \arg\max_{f} \mathbb{E}_{x \sim D_{train}}[g^*(x,f(x))]
    \end{aligned}
\end{equation}

Commonly, $g^*$ is either a (negative) loss or a (positive) reward.

Following the book "Pattern Recognition and Machine Learning" \cite{bishop} there are three learning paradigms common in machine learning: 
\begin{itemize}
	\item Supervised Learning
	\item Unsupervised Learning
	\item Reinforcement Learning
\end{itemize} .

\section{Supervised Learning}
\label{section:super_learn}
Supervised learning is a learning paradigm where the input-output pairs $(x,y) \in X \times Y$ are given, and the goal is to learn a function 
$f^*$ that maps inputs $x \in X$ to outputs $y \in Y$. The objective function is typically a loss function that measures the difference between 
the predicted output $f(x)$ and the true output $y$.

Formally, the signal or loss function $L = g^*$ is defined on a training set $D_{train} = {(x_i,y_i)}_{i=1}^n$ where $x_i \in X$ and $y_i \in Y$. Using the 
labels, it computes a loss between the output of a function $f(x_i)$ and a desired output $y_i$. The signal function is shaped such that optimizing $f$ for $L$ 
over $D_{train}$ should be aligned with optimizing $f$ for $g$ over $D$:

\begin{equation}
\begin{aligned}
f^* &= \arg\min_{f} \mathbb{E}_{(x,y) \sim D}[g(f(x), y)] \\
f^* &\approx f' = \arg\min_{f} \mathbb{E}_{(x,y) \sim D_{train}}[L(f(x),y)] = \arg\min{f} \frac{1}{n}\sum_{i=1}^n L(f(x_i),y_i)
\end{aligned}
\end{equation}


Supervised learning can be used for a variety of tasks, including classification, regression, and sequence prediction. In classification, 
the goal is to predict a discrete class label $y \in {1,2,\ldots,K}$ for a given input $x$. In regression, the goal is to predict a continuous 
output $y \in \mathbb{R}$ for a given input $x$. In sequence prediction, the goal is to predict a sequence of outputs $y_1,\ldots,y_T$ for a 
given input sequence $x_1,\ldots,x_T$ \cite[chapter~4]{bishop} \cite[chapter~5, chapter~6]{Goodfellow}.

\section{Unsupervised Learning}
\label{section:unsup_learn}
Unsupervised learning is a learning paradigm where the input data is unlabeled and the goal is to discover underlying patterns or structure in the data.
The objective function in unsupervised learning is typically not based on a pre-defined notion of correctness, but rather on measures of statistical
independence or data compression.\\

Unsupervised learning can be used for a variety of tasks, including clustering, density estimation, dimensionality reduction, and anomaly detection.
It is also often used as a pre-processing step in supervised learning, where the goal is to learn a representation of the input data that is more informative
for the downstream task \cite[chapter~9]{bishop} \cite[chapter~5]{Goodfellow}. 

\section{Reinforcement Learning}
\label{section:rl}

Reinforcement learning is a learning paradigm where an agent interacts with an environment and learns to take actions that maximize a 
cumulative reward signal. In this paradigm, the input $x$ is typically the state $s$ of the environment, the output $y$ is the action $a$ taken by the agent, 
and the the objective function $g$ is a reward signal $r$.

The agent learns a policy $\pi: S \to A$ that maps states to actions, in order to maximize the expected cumulative reward:

\begin{equation}
    \label{rl_objective}
    \begin{aligned}
    \pi^* &= \arg\max_{\pi} \mathbb{E}_{\tau \sim p(\tau | \pi)} \left[ \sum_{t=0}^T \gamma^t r_t \right] \\
    \approx \pi' = \arg\max_{\pi} \frac{1}{N} \sum_{i=1}^N \left[ \sum_{t=0}^{T_i} \gamma^t r_{i,t} \right]
    \end{aligned}
\end{equation}

where $\tau = (s_0, a_0, r_0, s_1, a_1, r_1, \ldots, s_T, a_T, r_T)$ is a trajectory, $p(\tau | \pi)$ is the probability of a trajectory 
under policy $\pi$, $\gamma \in [0,1]$ is a discount factor that controls the importance of future rewards, and $N$ is the number of trajectories sampled.

\subsection{Classification of Reinforcement Learning}
\begin{minipage}{\textwidth}
\begin{itemize}

    \item \textbf{Model type:}
    \begin{itemize}
    \item Model-based: the algorithm explicitly models the dynamics of the environment.
    \item Model-free: the algorithm learns a policy without explicitly modeling the environment.
    \end{itemize}
    \item \textbf{Action/Observation space:}
    \begin{itemize}
    \item Continuous: the action/observation space is continuous, typically represented by a real-valued vector.
    \item Discrete: the action/observation space is discrete, typically represented by a finite set of actions.
    \end{itemize}
    \item \textbf{Policy type:}
    \begin{itemize}
    \item On-policy: the algorithm learns the optimal policy while using the same policy to collect experience.
    \item Off-policy: the algorithm learns the optimal policy while using a different policy to collect experience.
    \end{itemize}
    \item \textbf{Reward type:}
    \begin{itemize}
    \item Dense reward: the agent receives a reward signal for every time step in the environment.
    \item Sparse reward: the agent receives a reward signal only for certain time steps in the environment.
    \end{itemize}
    \item \textbf{Time horizon:}
    \begin{itemize}
    \item Infinite horizon: the algorithm aims to maximize the expected sum of rewards over an infinite time horizon.
    \item Finite horizon: the algorithm aims to maximize the expected sum of rewards over a fixed time horizon.
    \end{itemize}
    \item \textbf{inference type:}
    \begin{itemize}
    \item Probabalistic inference: the algorithm samples actions according to a probability distribution.
    \item Deterministic inference: the algorithm chooses an action deterministicly.
    \end{itemize}
    \end{itemize}
\end{minipage}

\section{Markov Decision Processes}
Markov Decision Processes (MDPs) are a mathematical framework used to model decision-making processes that occur in a sequence. They provide a useful 
context to develop reinforcement learning algorithms.

An MDP is defined as a tuple $(S, A, T, R, \gamma)$, where:
\begin{itemize}
    \item $S$ is the set of possible states in the system
    \item $A$ is the set of possible actions that can be taken in each state
    \item $T$ is the transition function that specifies the probability of moving from one state to another given a certain action
    \item $R$ is the reward function that specifies the immediate reward obtained after taking a certain action in a certain state
    \item $\gamma$ is the discount factor that determines the importance of future rewards relative to immediate rewards
\end{itemize}

The transition function $T$ is defined as:

$$T(s, a, s') = \mathbb{P}(s_{t+1}=s' \mid s_t=s, a_t=a)$$

This function specifies the probability of moving from state $s$ to state $s'$ after taking action $a$. The reward function $R$ is defined as:

$$R(s, a) = \mathbb{E}[r_{t+1} \mid s_t=s, a_t=a]$$

This function specifies the expected immediate reward obtained after taking action $a$ in state $s$. The discount factor $\gamma$ is a 
parameter that determines the importance of future rewards relative to immediate rewards. It is typically a value between 0 and 1, where 0 means that 
only immediate rewards are important and 1 means that future rewards are equally important. \\

Importantly, the transition function $T$ satisfies the markov property. The Markov property characterizes stochastic processes where the future state of the 
process depends only on the current state and not on any past states. Formally, a stochastic process has the Markov property if, 
for all time steps $t$, states $s_t$, and actions $a_t$, the following condition holds:

\begin{equation}
    P(s_{t+1} \mid s_t, a_t, s_{t-1}, a_{t-1}, \ldots, s_1, a_1) = P(s_{t+1} \mid s_t, a_t)
\end{equation}

\section{Policy Gradient}
There are numerous ways to arrive at a policy that maximizes the reinforcment learning objective \ref{rl_objective}. For large or continuous action and observation spaces, a common approach is 
to parametrize the policy using a differentiable function. In this thesis we will be using neural networks for this task, as they have shown most promising results. The expectation over the cumulative 
reward $J = E[\sum_{t} \gamma^t r_t]$ with respect to a policy $\pi$ with parameters $\theta$ can be written as:

\begin{equation}
J(\theta) = \mathbb{E}_{\tau \sim p(\tau | \pi)} \left[ \sum_{t=0}^T \gamma^t r_t \right]
\end{equation}
where $\tau$ is a trajectory of state-action pairs, $p(\tau | \pi)$ is the probability of generating trajectory $\tau$ under policy $\pi$, and $\theta$ are the parameters of the policy $\pi$. 
$T$ can either be finite or infinite. If $T$ is infinite, $\gamma$ must be smaller then 1.

Our goal is to maximize $J(\theta)$ with respect to $\theta$. We can use the policy gradient theorem to compute the gradient of $J(\theta)$:
\begin{equation}
    \label{nabla_reinforce}
    \begin{aligned}
        \nabla_{\theta} J(\theta) = \mathbb{E}_{\tau \sim p(\tau | \pi)} \left[ \sum_{t=0}^T \nabla_{\theta} \log \pi(a_t|s_t;\theta) \cdot \sum_{t'=t}^T \gamma^{t'-t} r_{t'} \right]\\
        = \mathbb{E}_{\tau \sim p(\tau | \pi)} \left[ \sum_{t=0}^T \nabla_{\theta} \log \pi(a_t|s_t;\theta) \cdot  G_t\right]
    \end{aligned}
\end{equation}
With $G_t = \sum_{t'=t}^T \gamma^t' r_{t'}$ and $\pi(a_t|s_t;\theta)$ is the probability of taking action $a_t$ in state $s_t$ under policy $\pi$ with parameters $\theta$. 
$G_t$ represents the total discounted reward obtained after taking action $a_t$ in state $s_t$.

In practice we don't have access to the real expectation. Instead we sample n Trajectories and approximate the expectation.
A common choice is to sample one trajectory per update step. 
This finally gives us the gradient of the approximated expected reward with respect to the policy parameters $\theta$. We can use this gradient to update the parameters of the policy in the direction of steepest ascent. Specifically, we can use stochastic gradient ascent to update the policy parameters after observing a trajectory $\tau$:
\begin{equation}
    \label{reinf_update}
    \theta \leftarrow \theta + \alpha \nabla_{\theta} \log \pi(a_t|s_t;\theta) \cdot G_t
\end{equation}
where $\alpha$ is the learning rate. This update rule is used in the REINFORCE algorithm:

\begin{algorithm}[H]
    \SetAlgoLined
        \KwIn{Policy function $\pi_{\theta}(a|s)$, learning rate $\alpha$}
        \KwOut{Learned policy parameters $\theta$}
        Initialize policy parameters $\theta$;
        \While{not converged}{
            Sample a trajectory $\tau = (s_1, a_1, r_2, \dots, s_{T-1}, a_{T-1}, r_T, s_T)$ by following the policy $\pi_{\theta}$:\\
            \For{$t\leftarrow 1$ \KwTo $T$}{
                Compute the return starting from time $t$: \\
                $G_t = \sum_{k=t}^T \gamma^{k-t} r_k$ ;\\
                Compute the gradient of the log probability of taking action $a_t$ in state $s_t$: \\
                $\delta_{\pi(\theta)} = \nabla_{\theta}\log \pi_{\theta}(a_t|s_t)$; \\
                Compute the gradient estimate for this time step: \\
                $g_t \leftarrow G_t \delta_{\pi(\theta)}$;
                Update policy parameters using the gradient estimate:
                $\theta \leftarrow \theta + \alpha g_t$;
            }
        }
    \caption{REINFORCE algorithm with one trajectory per update}
\end{algorithm}
    
The algorithm collects 1 trajectory, computes the gradient estimate, and updates the policy parameters using stochastic gradient ascent. The algorithm continues to collect more trajectories and update the policy until convergence.


\section{Value Functions}

Value functions are a fundamental concept in reinforcement learning and are used to estimate the expected return of being in a particular state or taking a particular action. In this section, we will discuss the two main types of value functions in Markov decision processes (MDPs): state-value functions and action-value functions.
To clean up the notation, we will use upper case letters for functions, if the argument 
is implicit and lower case, if the argument is explicit.

\subsection{State-Value Function}

The state-value function $v_{\pi}(s)$ is the expected return when starting in state $s$ and following policy $\pi$ thereafter. It is defined as:

\begin{equation}
    v_{\pi}(s) = \mathbb{E}_{\tau \propto \pi}\left[\sum_{t=0}^{\infty} \gamma^t r_t \mid s_0 = s\right].
\end{equation}

$\mathbb{E}_{\tau \propto \pi}$ denotes the expected value under policy $\pi$.

The state-value function satisfies the Bellman equation, which expresses the relationship between the value of a state and the values of its successor states:

\begin{equation}
    \label{bootstrap_v}
    \begin{aligned}
        v_{\pi}(s) = \mathbb{E}_{\tau \propto \pi}\left[\sum_{t=0}^{\infty} \gamma^t r_t \mid s_0 = s\right] \\
        = \sum_{a \in \mathcal{A}} \pi(a \mid s) \left[ r(a,s)  \\
        + \gamma \sum_{s' \in \mathcal{S}} p(s' \mid s,a) \sum_{a' \in \mathcal{A}} \pi(a' \mid s') \left[ r(a',s') + \gamma ...\right] \right]\\
        = \sum_{a \in \mathcal{A}} \pi(a \mid s) \left[ r(a,s) +  \gamma \sum_{s' \in \mathcal{S}} p(s' \mid s,a) v_{\pi}(s')\right]
    \end{aligned}
\end{equation}

where $\mathcal{A}$ is the set of possible actions, $\mathcal{S}$ is the set of possible states, $p(s' \mid s,a)$ is the transition probability 
from state $s$ to state $s'$ under action $a$, $r(s,a,s')$ is the reward received when transitioning from state $s$ to state $s'$ under 
action $a$, and $\gamma \in [0,1]$ is the discount factor. This derivation assumes a discrete action space,
however it can be adapted for the continuous case by exchanging the summation with an integral over the probability density function.

\subsection{Action-State Value Function}

The action-state value function, also known as the Q-function, is the expected return when starting in state $s$, taking action $a$, and following 
policy $\pi$ thereafter. It is defined as:

\begin{equation}
    q_{\pi}(s, a) = \mathbb{E}_{\tau \propto \pi}\left[\sum_{t=0}^T \gamma^t r_t \mid s_0 = s, a_0=a\right]
\end{equation}

where $A_t$ is the action taken at time step $t$.

The action-state value function also satisfies the Bellman equation:

\begin{equation}
    \label{bmeq_q}
    q_{\pi}(s,a) = r(s,a) + \sum_{s' \in \mathcal{S}} p(s' \mid s,a) \left(\gamma \sum_{a' \in \mathcal{A}} \pi(a' \mid s') q_{\pi}(s',a')\right).
\end{equation}

Both value functions can be expressed in terms of the respective other value function using the Bellman equation:
\begin{equation}
    \label{q_from_v}
    q_{\pi}(s,a) = r(s,a) + \gamma \mathbb{E}_{s'\propto T(s,a,s')}\left[ v_{\pi}(s') \right]
\end{equation}

and 

\begin{equation}
    v_{\pi}(s) = \mathbb{E}_{a \propto \pi(\cdot|s)} \left[ q_\pi(s,a) \right]
\end{equation}



\section{Actor-Critic Algorithm}
One problem with REINFORCE is, that $G_t$ is computed from the rollout of a whole trajectory. As the policy changes with respect to the update step, this update 
rule is off policy with high variance. An idea is to approximate the expectation over $G_t$ using another neural network, so that an update can be made after every step, 
rather then after the whole trajectory. This approximation is referred to as a "critic" in the actor critic method.
\subsection{Critic}
In this section, we will derive an expression to aproximate $G_t$ by using an iterative update step with guaranteed convergence.
Revisiting \ref{nabla_reinforce}: 
\begin{equation}
    \begin{aligned}
        \mathbb{E}_{\tau \sim p(\tau | \pi)} \left[ \sum_{t=0}^T \nabla_{\theta} \log \pi(a_t|s_t;\theta) \cdot  G_t\right]\\
        = \mathbb{E}_{s_0, a_0, ... ,s_T, a_T \propto \pi} \left[
            \sum_{t=0}^{T}\nabla_{\theta} log \pi_{\theta}(a_t|s_t) \mathbb{E}_{r_t...r_T \propto \pi} \left[ G_t \right]
        \right]
    \end{aligned}
\end{equation}
, where 
\begin{equation}
    \begin{align}
        \mathbb{E}_{s_t, a_t, r_t...s_T, a_T, r_T \propto \pi} \left[ G_t \right] \\
        = r(s_t,a_t) + \gamma \mathbb{E}_{s_{t+1} \propto T(s_{t+1}|s_t, a_t)}\left[\mathbb{E}_{a_{t+1} \propto \pi(\cdot|s_t)}  \left[q_{\pi}(s_{t+1},a_{t+1})\right]  \right]
        = q_{\pi}(s_{t},a_{t}).
    \end{align}
\end{equation}
This follows the derivation as in \ref{bootstrap_v} for the $Q$-value in the continuous case.

Plugging in the defintion of the $Q$-value recursively gives us the Bellman equation for $Q$-values:
\begin{equation}
q_{\pi}(s_t, a_t) = \mathbb{E}_{\tau \propto p(\tau|\pi)} \left[ r(s_t,a_t) + \gamma q_{\pi}(s_{t+1}, a_{t+1})\right]
\end{equation}
for the sequence $\tau = (s_0,a_0, s_1, a_1...) \propto \pi$.\\
We want to find a policy $\pi^*$ for which the $Q$-function is maximal.
\begin{equation}
    Q^* = \max_{\pi} \left[\mathbb{E}_{\tau \propto \p(\tau|\pi)} \left[ r + \gamma Q_{\pi} \right]\right]
\end{equation}

Updateing the $Q$-Value greedily by applying the operator $\mathcal{B}_q$ is guaranteed to converge to the optimal $Q$-Value (see section \ref{opt_contr}):
\begin{equation}
    \begin{aligned}
        q_{i+1}(s_t, a_t) = (\mathcal{B}_q q_i)(s_t, a_t) \\
        =\mathbb{E}_{s_{t+1} \propto p(s_{t+1}|s_t, a_t)} \left[r(s_t, a_t) + \gamma  \max_{a'} \left[  q_i(s_{t+1}, a_{t + 1}) \right]\right].
    \end{aligned}
\end{equation}

We don't have access to the true expected value, but we can approximate it by using rollouts of our current policy, similar to the REINFORCE algorithm. As this introduces noice, 
in temporal difference learning (TD-learning) we define a learning rate $\alpha$ to smooth the update of the $Q$-Value:
\begin{equation}
    q_{i+1}(s_t, a_t) = q_i(s_t, a_t) + \alpha\left[r(s_t, a_t) + \gamma \left( \max_{a'} q(s_{t+1}, a')\right) - q_i(s_t, a_t) \right]
\end{equation}
It is prooven that for a sufficiently small $\alpha$ this update rule will converge to the optimum $Q$-value \cite{Watkins1992}. From this expression we can define a parametric update step by minimizing 
the RSME $J_Q(\theta_{i})$ between $Q_{\pi_{\theta_{i-1}}}' = \mathcal{B}_q Q_{\pi_{\theta_{i-1}}}$ and $Q_{\pi_{\theta_i}}$:
\begin{equation}
    \begin{aligned}
        J_Q(\theta_{i}) = \mathbb{E}\left[\left( Q_{\pi_{\theta_{i-1}}}' - Q_{\pi_{\theta_i}} \right)^2\right]\\
        = \mathbb{E}_{s_t, a_t}\left[\left( r(s_t, a_t) + \gamma \left( \max_{a'} q_{\theta_{i-1}}(s_{t+1}, a')\right) - q_{\theta_{i}}(s_t, a)\right)^2\right]
    \end{aligned}
\end{equation}

We can now use this error to update the parameters of the $Q_{\theta}$-function:
\begin{equation}
    Q(\theta_{i+1}) : \theta_{i+1} = \theta_{i} - \alpha \nabla J_Q(\theta_{i})
\end{equation}
\subsubsection{Optimality}
\label{opt_contr}

We will cover the proof of the existens and uniqueness of $v^*$. Then we will show it's optimality. The respective proofs for $q^*$ are analogous and can be found for example in \cite{Watkins1992}.
Parts of this derivation follows \cite{optimality}.\\ \\

\textbf{Definition:}\\
Let $(X, d)$ be a complete metric space. A a map $T:X \rightarrow X$ is called a contractor if there exists a factor $\gamma \in [0, 1)$ s.t.
\begin{equation}
    d(T(x), T(y)) \leq \gamma \cdot d(x,y) | x,y \in X.
\end{equation}
\\ \\

\textbf{Banach Fixed Point Theorem:}\\
Let (X,d) be a complete metric space and $T:X \leftarrow X$ a contractor. 
Then $T$ has a unique fixed point $x^* \in X$, s.t. $T(x^*) = x^*$. The sequence $lim_{n \leftarrow \infty}T^n(x)$ converges to $x^*$.\\ \\

\textbf{Value Iteration}\\
Let $X : v(s) \in \mathcal{R}$ and $d = ||X||_{\infty} = \max|X|$. \\
The value iteration operator $\mathcal{B}$ is defined as:
\begin{equation}
    (\mathcal{B}v)(s) = \max_{a \in \mathcal{A}} \left[  r(s,a) +  \gamma \sum_{s' \in \mathcal{S}} p(s' \mid s,a) \left( v(s')\right) \right]
\end{equation}
\textbf{Assertion}\\

$\mathcal{B}$ is a contrator under the $\infty$-norm.\\ \\

\textbf{Proof:}\\
Let $V_1$, $V_2$ be two value functions $V : S \times A \to \mathcal{R}$, $0 \leq \gamma < 1$ and $\mathcal{B}$ the value iteration operator, then:
\begin{equation}
    \label{contractor}
||\mathcal{B}V_1(s) - \mathcal{B}V_2(s)||_{\infty, s} = 
\end{equation}
\begin{equation}
    \left|\left| \max_a \left[ r(s,a) + \gamma \sum_{s'} p(s' \mid s,a) v_1(s')\right] - \max_{a'}\left[ r(s,a') + \gamma \sum_{s'} p(s' \mid s,a') v_2(s') \right] \right|\right|_{\infty, s}
\end{equation}
\begin{equation}
    \leq \left|\left|\max_a \left[ r(s,a) + \gamma \sum_{s'} p(s' \mid s,a) v_1(s') - r(s,a) - \gamma \sum_{s'} p(s' \mid s,a) v_2(s')\right]\right|\right|_{\infty, s}
\end{equation}
\begin{equation}
= \gamma \left|\left|\max_a \left[ \sum_{s'} p(s' \mid s,a) (v_1(s') -v_2(s'))\right]\right|\right|_{\infty, s}
\end{equation}
\begin{equation}
\leq \gamma \left|\left|\max_a \left[ \sum_{s'} p(s' \mid s,a) \cdot \max_{s''} (v_1(s'') -v_2(s''))\right]\right|\right|_{\infty, s}
\end{equation}

\begin{equation}
= \gamma \left|\left|\max_{s''} \left[ (v_1(s'') - v_2(s''))\right]\right|\right|_{\infty, s}
\end{equation}

\begin{equation}
= \gamma \left|\left|(v_1(s'') - v_2(s''))\right|\right|_{\infty, s''}
\end{equation}

\begin{equation}
= \gamma \left|\left|(v_1(s) - v_2(s))\right|\right|_{\infty, s}.
\end{equation}

This leaves us to show, that the fix point $V_n$ under the value iteration operator is also the maximum of the value function $V$.\\
$\mathcal{B}$ is a monoton increasing operator on $V$: 
\begin{equation}
    \begin{align}
        \mathcal{B}v(s) = \max_{a \in \mathcal{A}} \left[ r(s,a) + \gamma  \sum_{s' \in \mathcal{S}} p(s' \mid s,a) v(s')\right]\\
        \geq r(s,a') + \gamma  \sum_{s' \in \mathcal{S}} p(s' \mid s,a') v(s') |a' \in A.
    \end{align}
\end{equation}
As $V_n$ is monoton increasing and converges to a unique value, it will approach it's supremum. 
For any practical application, this means it converges to it's maximum.

\subsection{Actor}
In the derivation of the Bellman operator update rule $\mathcal{B}_q$ we used the $\max$-operator to choose the best action to ensure a monoton increasing value estimate. In practice this is again unfeasable 
for large or continuous action spaces, so we want a parametrized $\pi$ that computes an action. Note that we only need a monoton increasing value estimate to guarantee asymptotic convergence. One way to 
achieve this is by minimizing the Kullback-Leibler (KL) divergence between the normalized $Q$-Value and the policy $\pi$. 
\begin{equation}
    \label{kl-obj}
    \begin{aligned}
        J_\pi(\phi_{i}) = D_{KL} \left( \pi_{\phi_{i}}(\cdot|s_t) || \frac{\exp(Q^{\pi_{\phi_{i-1}}}(\cdot|s_t))}{\underset{a}{\sum}  \exp(Q^{\pi_{\phi_{i-1}}}(a|s_t))} \right) \\
        \pi_{new} = \underset{\pi' \in \Pi}{\arg\min} J_\pi(\phi_{i})
    \end{aligned}
\end{equation}
with the update rule
\begin{equation}
    \pi_{\phi_{i}} : \phi_{i} = \phi_{i-1} - \alpha \nabla J_\pi(\phi_{i-1})
\end{equation}
, where $\Pi$ represents the class of all policies that can be parametrized using the respective architecture. \\
The KL divergence is a measure of the alikeness of two distributions $P(x)$ and $Q(x)$ over the same sample space $X$:
\begin{equation}
    \label{KL}
    \mathrm{KL}(P\|Q) = \int_{x\in X} P(x) \log \frac{P(x)}{Q(x)} \mathrm{d}x
\end{equation}
It is non negative, asymmetric in the distributions and $\mathrm{KL}(P\|Q) = 0$ if and only if $P = Q$. In addition to that, it can be differentiated, making it 
a common choice in deep learning applications.\\

In the paper "Soft Actor-Critic" by Haarnoja et al. \cite{haarnoja2018soft}, 
Lemma 2 prooves the optimality and convergence of this approach. The paper deals with a slightly altered stochasitc formulation of the actor critic paradigm, however the Lemma is generally applicable.

\subsection{Soft Actor-Critic}
\label{SAC}
In the paper "Soft Actor-Critic" from Tuomas Haarnoja, et al. \cite{haarnoja2018soft} anew approach for exploration is introduces. Exploration vs. exploitation 
is a fundamental trade-off in reinforcement learning. On the one hand, an agent must exploit the current knowledge to maximize the cumulative reward it receives 
over the course of its interactions with the environment. On the other hand, the agent must explore new actions and states to improve its knowledge and avoid getting stuck in suboptimal policies. 
The balance between exploration and exploitation is crucial for the success of RL algorithms, and different methods aim to find a good compromise between them.

The soft actor-critic (SAC) algorithm extends the AC method by introducing a new objective that encourages the policy to be more stochastic. Specifically, the SAC algorithm maximizes a modified 
version of the expected cumulative reward that takes into account both the expected reward and the entropy of the policy. The modified objective is given by:

\begin{equation}
J(\pi) = \mathbb{E}_{s_t \sim \mathcal{D}, a_t \sim \pi}[r(s_t, a_t) + \alpha \mathcal{H}(\pi(\cdot|s_t))]
\end{equation}

where $\mathcal{D}$ is the replay buffer, $r(s_t, a_t)$ is the immediate reward for taking action $a_t$ in state $s_t$, and $\mathcal{H}(\pi(\cdot|s_t))$ is the entropy of the policy. The entropy 
encourages the policy to be more stochastic and explore different actions, while the temperature parameter $\alpha$ controls the trade-off between exploration and exploitation.

To update the policy, the SAC algorithm uses a state-action value estimate $Q$ which is learned using a state value estimate $V$ similar to \ref{q_from_v}, 
but with an extra term for the entropy of the policy. $V$ is learned with the minimum of two separate critics $Q_1^\theta(s,a)$ and $Q_2^\phi(s,a)$ 
to minimize overestimation:

\begin{equation}
    V_{\psi}(s_t) = \mathbb{E}_{a_t \sim \pi_{\phi(a_t|s_t)}}[Q_\text{min}(s_t, a_t) - \alpha \log \pi_{\phi(a_t|s_t)}]
\end{equation}

Using this, the critics are trained with typical TD:

\begin{equation}
    \mathcal{J}(Q_{\theta_i}) = \mathbb{E}_{(s_t, a_t, r_t, s{t+1}) \sim \mathcal{D}}[(Q_{\theta_i}(s_t,a_t) - y_t)^2]
\end{equation}

where $y_t = r(s_t, a_t) + \gamma \mathbb{E}_{s_{t+1} \sim p}[V(s_{t+1})]$ is the target value. The policy is updated to minimize the KL-Divergence to the normalized $Q$-Values as described in \ref{kl-obj}:
\begin{equation}
    \label{sac_pol_obj}
    J_\pi(\phi_{i}) = D_{KL} \left( \pi_{\phi_{i}}(\cdot|s_t) || \frac{\exp(Q_{min, {\theta_i}}(\cdot|s_t))}{\underset{a}{\sum}  \exp(Q_{\theta_i}(a|s_t))} \right).
\end{equation}

The policy is inherently stochastic and returns an action according to a spherical gaussian distribution with noise vector $\epsilon_t$, learned mean and variance $a_t = f_{\phi}(\epsilon_t;s_t)$. 
To approximate the derivative of \ref{sac_pol_obj} given limited sample size, the reparametrisation trick is used:

\begin{equation}
    \begin{align}
        \nabla_{\phi}J_\pi(\phi_{i}) \approx \mathbb{E}_{\mathcal{D}} [\nabla_{\phi} log (\pi(\phi_{i})(a_t, s_t))\\
        + \left( \nabla_{a_t} log (\pi(\phi_{i})(a_t, s_t)) - \nabla_{a_t} Q_{\theta_i}(a_t, s_t) \right) \nabla_{\phi} f_{\phi}(\epsilon_t;s_t)]
    \end{align}
\end{equation}

To summarize, SAC introduces a parameter $\alpha$ with which the trade off between exploration and exploitation can be tuned. All algorithms discussed in this thesis will use a stochastic 
actor and entropy regularisation.
\subsection{PPO}
A motivation to develop Actor-Critic methods was the high variance of the cumlative reward $G_t$ used in the update step of the REINFORCE algorithm 
\ref{reinf_update}. To further improve the stability of the gradient update, a common approach is to use the advantage function $A(s,a)$ instead of the 
$Q$-value: 
\begin{equation}
    A(s,a) = Q(s,a) - V(s).
\end{equation}
Intuitively, the advantage $A$ is a measure for how good a specific action $a$ is compared to the average action taken by the current policy. 
Plugging in the advantage estmiator into \ref{reinf_update} gives the update rule:
\begin{equation}
    \nabla_{\theta} J(\pi_{\theta}) = \mathbb{E}_{(a, s, t) \propto \pi_{\theta}}[\nabla_{\theta} log(\pi_{\theta}(a|s))A(s,a)]
\end{equation}
with the corresponding loss:
\begin{equation}
    \label{LPG}
    L^{PG}(\theta) = \mathbb{E}_{(a, s, t) \propto \pi_{\theta}}[-log(\pi_{\theta}(a|s))A(s,a)].
\end{equation}
Most common implementations of PPO use a stochastic policy, like in SAC, with the corresponding loss:
\begin{equation}
    \label{PPO_Loss_Reg}
    L^{PG}(\theta) = \mathbb{E}_{(a, s, t) \propto \pi_{\theta}}[-log(\pi_{\theta}(a|s))A(s,a)+\alpha log(\pi_{\theta}(a|s))].
\end{equation}
The paper "Proximal Policy Optimization Algorithms" (PPO) \cite{PPO} introduced by Schulman et al. 
builds upon the ideas of Trust Region Policy Optimisation (TRPO) \cite{TRPO}. In TRPO, instead of the objective \ref{LPG}, a "surrogate" objective $L^{TRPO}$ is optimized 
with respect to a KL divergence constraint:
\begin{flalign}
        \text{optimize } L^{TRPO}(\theta) = \mathbb{E}_{(a, s, t) \propto \pi_{\theta}} \left[\frac{\pi_{\theta}(a|s)}{\pi_{\theta_{old}}(a|s)} A(s,a)\right] \\
        \text{subject to\ \ \ \ \ \ \ \ \ \ \ \ } \mathbb{E}_t[\text{KL}[ \pi_{\theta_{\text{old}}}(\cdot | s_t), \pi_{\theta}(\cdot | s_t)] ] \leq \delta
\end{flalign}
The core idea is to place a constraint on the distance between the new policy and the old policy. By doing so, 
too large update steps to the policy should be prevented, resulting in a more stable learning behaviour. Note that the logarithm is monton and 
$\pi_{\theta_{old}}$ is constant w.r.t. $\pi_{\theta}$. That means optimizing the unconstrained TRPO objective opimizes $L^{PG}$, thus the optimality proof of 
$L^{PG}$ holds for $L^{TRPO}$. A more general proof considering the stochasticity of the policy update is provided in the paper Trust Region Policy Optimization, 
theorem 1 \cite{TRPO}. \\

In TRPO, ensuring the KL constraint requires a second order approximation to it. PPO is an on-policy algorithm that uses only first order approximations, while 
also making sure that the policy update does not change the policy too much. Instead of the hard KL constraint, it clips the gradient of the policy update within 
an $\epsilon$ region around the old policy. This way, chainging the policy further does not improve the objective.\\
Let
\begin{equation}
    r_t = \frac{\pi_{\theta}(a_t|s_t)}{\pi_{\theta_{old}}(a_t|s_t)}
\end{equation}

\begin{equation}
        L^{CLIP}(\theta) = {E}_t \left[ \min \left(r_t(\theta) {A}_t, \text{clip} \left(r_t(\theta), 1-\epsilon, 1+\epsilon \right) {A}_t \right) \right].
\end{equation}
The algorithm aims to minimize $L^{CLIP}(\theta)$ by doing gradient update steps:
\begin{equation}
    \theta_{i+1} = \theta_{i} + \alpha \nabla_{\theta} L^{CLIP}(\theta_i)
\end{equation}
The PPO algorithm follows the general actor critic scheme with an altered objective function.

\begin{algorithm}[h]
    \SetAlgoLined
    \begin{algorithmic}
    \For{$\text{iteration}=1,2,\dots$}{ \\
        \State Run policy $\pi_{\theta_\text{old}}$ in environment for $T$ timesteps\\
        \State Compute advantage estimates $\hat{A}_1, \dots, \hat{A}_T$\\
        \State Optimize surrogate $L$ with respect to $\theta$, with $K$ epochs and minibatch size $M \leq T$\\
        \State $\theta_\text{old} \gets \theta$
        }
        
    \end{algorithmic}
    \caption{PPO, Actor-Critic Style}\label{ppo}
\end{algorithm}

Overall, PPO is a robust and effective algorithm for deep reinforcement learning, and has been successfully applied to a wide range of environments and tasks.

\subsection{TQC}
Controlling Overestimation Bias with Truncated Mixture of Continuous
Distributional Quantile Critics \cite{TQC_Paper} is a paper describing an off policy algorithm that aims to minimize the over estimation bias.\\
Emperically, the learned $Q$ function tends to overestimate the actual $Q$ value. We already saw one way to counter this bias by using the minimum value of 
two critics in the SAC algorithm \ref{SAC}. In the paper \cite{TQC_Paper} the authors cite a formal derivation of the over estimation bias by 
Thrun and Schwartz (1993) \cite{thrun1993issues} stating:
\begin{equation}
    \begin{align}
        \max_a Q(s,a) = \max_a \mathbb{E}_{U}\left[Q(s,a) + U(a)\right] \leq \mathbb{E}_{U}\left[\max_a \left{Q(s,a) + U(a)\right}\right]
    \end{align}
\end{equation}
, where $U(a)$ is action dependent random noise with zero mean. \\
The main idea of the paper is to learn the probability distribution of the $Q$ value, rather then approximating the expectation of the $Q$ value. 
It is then possible to drop the "right tail", that is the highest estimates for the $Q$ value, to lower the estimate.\\
Formally, the goal is to approximate the random variable $Z^\pi(s,a):=\sum_{t=0}^\infty\gamma^tR(s_t, a_t)$, rather then 
$Q^\pi  := \mathbb{E}[Z^\pi(s,a)] = \mathbb{E}[\sum_{t=0}^\infty\gamma^tR(s_t, a_t)]$. This is done using quantile regression loss.\\
The kth $M$-quantile of a distribution with random variable $X$ is defined as the the value $x$ for which the comulative distribution function crosses $k/q$:
\begin{equation}
    Pr[X \leq x] \leq k/M
\end{equation}
with $M$ total quatiles. The mean of all quantiles is the mean of the distribution.\\
Quantile regression is used to learn the values of the $M$ quantiles:
\begin{equation}
    \label{rho}
    \begin{align}
    L^{\tau_m}_{QR}(\theta^m) := \mathbb{E}_{\tilde{Z}\sim Z}\left[\rho_{\tau_m}(\tilde{Z}-\theta^m)\right],\\
    \quad \text{where} \quad \rho_{\tau_m}(u) = u({\tau_m} - I(u < 0)), \quad \forall u \in \mathbb{R}.
    \end{align}
\end{equation}
with ${\tau_m = \frac{2m-1}{2M}}, m \in [1, ...., M]$.\\
Let $\theta^m(s,a)$ be the approximation of the $m$-th quatile of $Q(s,a)$. The probability distribution $Z(s,a)$ can be approximated using $\theta^m$:
\begin{equation}
    Z_{\theta}(s,a) (v) = \frac{1}{M} \sum_{m=1}^M 1_{\theta^m(s,a) \geq v} | v\in\mathcal{R}.
\end{equation}
The temporal difference target can then be written as:
\begin{equation}
    y_i(s,a) := r(s,a) + \gamma \mathbb{E}_{s' \propto \mathcal{D}, a' \propto \pi(\cdot|s')}[\theta^i(s',a') - \alpha \log \pi_{\varphi}(a'|s')]
\end{equation}
with the tagret distribution
\begin{equation}
    Y(s,a)(v) = \frac{1}{N} \sum_{i=1}^N 1_{y_i(s,a) > v} | v \in \mathcal{R}
\end{equation}
, where only the lowest N quantiles were used. \\
Finally this expression leads us to the loss functino for the quantiles $\theta_m^{\psi}$ of Y:
\begin{equation}
    L_k(s,a;\psi) = \frac{1}{kN}\sum_{m=1}^{M}\sum_{i=1}^{kN}\rho^H_{\tau_m} \left(y_i(s,a) - \theta_m^{\psi}(s,a)\right)
\end{equation}
, where $\psi$ deontes the parametrisation of $\theta$. Note that instead of the function $\rho_{\tau_m}(u)$ as described in \ref{rho}, they used the 
1-Huber loss, which emperically got better results. \\
From this we can find the entropy regularized policy loss function $J$:
\begin{equation}
    J_\pi(\phi) = \mathbb{E}_{D \propto \pi}\left[\alpha\log\pi{\phi}(a|s) - \frac{1}{M}\sum_{m=1}^{M}\theta_m^{\psi}(s,a)\right]
\end{equation}
, similar to the PPO loss \ref{PPO_Loss_Reg}. In the paper, they use N critics $\psi_n$. To keep the notation clean, I derived the main ideas using only one. An 
overview over the algorithm with N critics can be found in \ref{algo:TQC}.

\begin{algorithm}
    \caption{TQC Algorithm [\ref{algo:TQC}]}
    \label{algo:TQC}
    \begin{algorithmic}
        \State Initialize policy $\pi_{\phi}$, critics $Z_{\psi_n}$, $Z_{\psi_n}$ for $n \in [1..N]$\\
        \State Set replay $D = \emptyset$, $\beta = 0.005$\\
        \For{each iteration}{
            \For{each environment step, until done}{
                \State Collect transition $(s_t, a_t, r_t, s_{t+1})$ with policy $\pi_{\phi}$ \\
                \State $D \leftarrow D \cup {(s_t, a_t, r_t, s_{t+1})}$\\
            \EndFor}
            \For{each gradient step}{
                \State Sample a batch from the replay $D$\\
                \State $\phi \leftarrow \phi - \lambda{\pi} \hat{\nabla}{\phi}J{\pi}(\phi)$\\
                \State $\psi_n \leftarrow \psi_n - \lambda_Z \hat{\nabla}{\psi_n}J_Z(\psi_n), \text{for } n \in [1..N]$\\
                \State $ \overline{\psi_n} \leftarrow \beta \psi_n + (1 - \beta)\overline{\psi_n}, \text{for } n \in [1..N]$
            \EndFor\\}
        \EndFor\\}
        \State \textbf{return} policy $\pi_{\phi}$, critics $Z_{\psi_n}$
    \end{algorithmic}
\end{algorithm}
TQC achieves state of the art performance in a wide variety of chllenging tasks, like robotic maipulation in the open AI gym environment.

\section{Imitation Learning}
Imitation learning is a type of supervised learning in which an agent learns a policy by observing expert demonstrations, 
rather than by trial and error exploration \cite{IL}. In this approach, the data distribution seen during test time is dependent on the learned policy, 
breaking the i.i.d. (independent and identically distributed) assumption typically made in supervised learning. This means that the performance of the 
learned policy can be severely impacted by distributional shifts between the training data and the test data. \\ \\

Formally, imitation learning can be defined as the task of learning a policy $\pi^*$ that maximizes the expected discounted sum over rewards during test time, given a dataset of expert demonstrations $D = {s_1, a_1, ..., s_T, a_T}$:

\begin{equation}
    \pi^* = \underset{\pi}{\text{argmax}}\left[\mathbb{E} [\sum_{t=0}^{\infty} \gamma^t r(s_t, a_t)]\right]
\end{equation}

where $s_t$ is the state at time step $t$, $a_t$ is the action taken by the agent at time step $t$, $r(s_t, a_t)$ is the reward received for taking action $a_t$ in state $s_t$, and $\gamma \in [0,1]$ is the discount factor.

Returning to our general learning paradigm \ref{general_learning_paradigm}, $g^*$ has access to expert Data $D_{\text{expert}}$, but the challenge of imitation learning 
opposed to the general supervised learning paradigm is, that the data disctibution during test time depends on the learned policy $\pi_{\text{imitaion}}$:
\begin{align}
    \mathbb{E}_{\mathcal{D} \propto \pi_{\text{imitaion}}}[\sum_{t=0}^{\infty} \gamma^t r(s_t, a_t)] \neq \mathbb{E}_{\mathcal{D} \ propto \pi_{\text{expert}}}[\sum_{t=0}^{\infty} \gamma^t r(s_t, a_t)] \quad \text{in general}
\end{align}
\\ \\
The goal of imitation learning is to learn a policy that performs as well as the expert demonstrations in a new, unseen environment, despite 
the potential for distributional shift. This approach has been applied successfully in a variety of domains, including robotics, video games, 
and natural language processing.

\subsection{Behavioral Cloning}
Behavioral cloning is a popular technique used in imitation learning, where the goal is to learn a policy that mimics the expert behavior seen in 
the training data. In behavioral cloning, the policy is learned by minimizing the difference between the actions taken by the expert and the 
actions predicted by the learned policy. Formally, let $a_{\text{expert}}$ be the action taken by the expert and $a_{\text{imitation}}$ be the action 
predicted by the learned policy. The goal is to minimize the expected distance between the two actions:

\begin{equation}
\min_{\theta} \mathbb{E}{(s, a{\text{expert}}) \sim D_{\text{expert}}} [d(a_{\text{expert}}, a_{\text{imitation}})]
\end{equation}

where $d(\cdot, \cdot)$ is a distance metric, and $\theta$ are the parameters of the learned policy. The distance metric could be, for example, the mean 
squared error or the cross-entropy loss.


\subsection{Behavioral Cloning for Stochastic Policies}
Behavioral cloning for stochastic policies is a variant of behavioral cloning that is suited for learning policies that output probability distributions over actions. In this setting, the goal is to maximize the probability of the learned policy selecting the action proposed by the expert, rather than simply minimizing the distance between the expert and learned actions. 
A common choice to achieve this is to minimze the KL divergence between the expert policy and the trained policy.

Formally, let 

\begin{center}
    \pi_{\text{expert}}(a|s) = 
        \begin{cases}
            1 \ |\ (a|s)\in \mathcal{D}_{\text{expert}}\\
            0, \text{else}
        \end{cases}

\end{center}

be the probability of the expert selecting action $a$ in state $s$, and let $\pi_{\theta}(a|s)$ be the probability of the learned policy. Then the KL divergence can be approximated by
\begin{equation}
    D_{KL}(\pi_{\text{expert}} || \pi_{\theta}) \approx \sum_{a,s \in \mathcal{D}_{\text{expert}}} \pi_{\text{expert}}(a,s) log\left(\frac{\pi_{\text{expert}}(a,s)}{\pi_{\theta}(a|s)}\right)
\end{equation}

Differentiating the KL divergence with respect to $\theta$ gives:
\begin{equation}
    \nabla_{\theta} D_{KL}(\pi_{\text{expert}} || \pi_{\theta}) \approx \nabla_{\theta} (-log\left({\pi_{\theta}(a|s)}\right))
\end{equation}
, as $\pi_{\text{expert}}(a,s)$ does not depend on $\theta$.

\subsection{Inverse Reinforcement Learning}

\section{Partially Observable Markov Decision Processes}

\subsection{Symmetry}