% !TEX root = ../main.tex

\chapter{Methodology}
\label{chapter:Methodology}
In the following chapter, we will first develop the \ac{avc} algorithm defined on \ac{sopomdp}s and then relax 
the constraint to the \ac{mdp} setting. To indicate an observation after each step, we will refer to the \ac{mdp} setting as continued observation 
and call the \ac{avc} algorithm for \ac{mdp}s \ac{coavc}.\\

The \ac{avc} architecture consists of an actor and a critic, similar to actor critic methods. The main difference is, that \ac{avc} uses the whole sequence of 
actions, rather than on the current observation and action, to plan the current step.\\ 

Formally, let $s_{1}$ be the state at time step $1$ and let $a_{1:T}$ be an action sequence with $T$ actions. 
We define a whole sequence actor as a model, that takes $s_{1}$ as input and returns actions 
\begin{equation}
    \label{eq:def_wsa}
    \pi(s_{1})(t) = a_{1:T}(t).
\end{equation}
Similar, we define a whole sequence critic as: 
\begin{equation}
    \label{eq:def_wsc}
    C_{s_1, a_{1:T}}(t) = \mathbb{E}_{s_t \propto \tau_{s_1, a_{1:T}}}\left[r(a_t, s_t)\right],
\end{equation}
where $t$ is the time step. In other words, the actor and the critic return $T$ values, for which the $t'$-th value is defined by the 
equations above.\\

In the following chapter, we will first justify using a transformer encoder architecture for our neural network architecture. 
Then we will motivate the whole sequence formulation, derive the updates for the critic and the 
actor and provide analysis of optimality and policy improvement of our method. 

\section{Single Observation Partially Observable Markov Decision Process}
We identify two main challenges in dealing with \ac{sopomdp}s. \\
The first challenge is the curse of dimensionality from the large preimage of the policy, given it has to keep track 
of the past. \\
The second challenge is unstable update behavior of actor critic methods which comes from update errors by the policy gradient. This problem 
is specifically pronounced in settings with highly non-smooth feedback from the environment, like in our case, given sparse rewards 
and no further observations.
\subsection{Curse of Dimensionality}
\label{COD_AC}
As introduced in Section \ref{COD}, the evidence for a model scales approximately inversely proportional to the exponent of the dimension on which it acts. In \ac{mdp}s, the 
actor $\pi$ acts on the observation space and returns an action from the action space. A probabilistic actor defines a probability distribution on the set 
of actions $\mathcal{A} \in \mathcal{R}^m$ and observations $\mathcal{O} \in \mathcal{R}^n$ as $\pi(a|o) = \frac{p(a,o)}{p(o)}$. This also holds for 
deterministic actors with integer probabilities $\pi(a|s) \in \{0,1\}$.
Thus the dimension of the preimage of the actor is $n \times m$ with $\pi:\mathcal{R}^m \times \mathcal{R}^n \rightarrow \mathcal{R}$.\\

Similarly, the critic $Q$ usually defines a function on $\mathcal{A}$ and $\mathcal{O}$ to the cumulative discounted expected value 
$Q:\mathcal{R}^m \times \mathcal{R}^n \rightarrow \mathcal{R}$\\
We have shown in Section \ref{POMDP}, that in a \ac{pomdp}, the observation space grows linear in time, as the belief state depends on the history: 
$$\pi(a| b_t(o_{1:t}, a_{1:t})):\mathcal{R}^{m \times T} \times \mathcal{R}^{n \times T} \rightarrow \mathcal{R},$$
$$Q(a, o_{t}| b_{t-1}(o_{1:t-1}, a_{1:t-1})):\mathcal{R}^{m} \times \mathcal{R}^{n} \times \mathcal{R}^{m \times T-1} \times \mathcal{R}^{n \times T-1} \rightarrow \mathcal{R}.$$
Together with the result from \ref{COD} we conclude, that we have 
exponentially less evidence with sequence length $T$ for our model in the \ac{pomdp} setting.\\

To counter this problem, we want to get rid of the time dependency of the belief state. 
We make the observation, that our case with the single observation at the beginning is a special case of a \ac{pomdp}.\\

Usually, 
we need to know all previous observations, as they can give us additional information about the believe state. However, as we don't have subsequent observations 
after the initial, we only need the actions of the trajectory to update our belief $b_t$. Formally, as introduced in Equation \ref{pomdp_bayes}, the optimal believe state 
update in \ac{pomdp}s using Bayes theorem is given by:
\begin{equation}
    b_{t+1}(s') = \frac{\mathcal{Z}(s', o_{t+1}) \sum_{s \in \mathcal{S}} \mathcal{T}(s, a_t, s') b_t(s)}{\sum_{s'' \in \mathcal{S}} \mathcal{Z}(a_t, s'', o_{t+1}) \sum_{s \in \mathcal{S}} \mathcal{T}(s, a_t, s') b_t(s)}.
\end{equation}
In our case, we can rewrite this as:
\begin{equation}
    b_{t+1}(s') = \frac{\mathcal{Z}(s', o_{1}) \sum_{s \in \mathcal{S}} \mathcal{T}(s, a_t, s') b_t(s)}{\sum_{s'' \in \mathcal{S}} \mathcal{Z}(s'', o_{1}) \sum_{s \in \mathcal{S}} \mathcal{T}(s, a_t, s') b_t(s)}.
\end{equation}
$\mathcal{Z}(s', o_{1})$ is constant in $t$, so we get:
\begin{equation*}
    b_{t+1}(s') \propto \sum_{s \in \mathcal{S}} \mathcal{T}(s, a_t, s') b_t(s),
\end{equation*}
\begin{equation}
    b_{1}(s') \propto \mathcal{Z}(s', o_{1})
\end{equation}
If we assume a deterministic policy $\pi(a|b) \rightarrow a_{\pi}(b)$, the update step can be written as:
\begin{equation}
    b_{t+1}(s') \propto \sum_{s \in \mathcal{S}} \mathcal{T}(s, a_{\pi}(b_t), s') b_t(s).
\end{equation}
By using this update step recursively until we get to $b_{1}$, we see, that the believe state at time $t$ only depends on the initial observation $o_1$,
the transition probability $\mathcal{T}$, the policy $\pi$ and the number of recursion steps $t$. Using this insight, we can rewrite the belief state at time step $t$ like:
\begin{equation}
    b_{t+1}(s') \propto \mathcal{Z}(s_1, o_{1}) \mathcal{T}(s_1, s', t)_{\pi},
\end{equation}
where $\mathcal{T}(s_1, s', t)_{\pi}$ indicates that the normalized transition function depends on the time step $t$ and policy $\pi$.\\
With this formulation, we find that the preimage of the policy given deterministic actions can be reduced to:
$$\pi(a|b_t) = \pi(a|o_1, t): \mathcal{R}^m \times \mathcal{R}^{n} \times \mathcal{R} \rightarrow \mathcal{R}.$$
We call this a strong inductive belief state bias for the \ac{sopomdp}, as it forces the assumption of a deterministic and constant policy $\pi$, so that the prior action sequence is known. 
Here we see, 
why this setup is specifically interesting for planning. Planning ahead fulfills the same constraint, namely that we can plan the actions, but we don't get feedback of the 
observations we would receive, if we acted out a planned sequence. \\

In our method, we implement a weak inductive bias, which is motivated by this observation. We use a transformer encoder type architecture, whose inputs are 
the repeated initial observation plus a positional encoding. To generate the action sequence, we use a single pass through the network. While 
information between time steps can be exchanged through the attention mechanism, earlier actions are not part of the input of later actions. 
We will discuss the architecture in more details in the coming sections. For now, we only want to motivate the single pass through 
our network to generate an action sequence, rather then generating it autoregressively like recurrent methods would.\\ 

Another benefit of computing the whole trajectory in one pass is that the \ac{iid} input data 
assumption underlying supervised learning methods is not broken. The assumption states, that we don't expect a different data distribution during inference time 
then during train time, which generally does not hold for \ac{il}, as the distribution of data seen during inference depends on the policy. 
However, by computing the whole trajectory at once, the input data is distributed according to the initial state of the \ac{pomdp}, which is independent of 
the policy, thus the \ac{iid} assumption holds.\\

\section{Approximation Error}
\label{inference_time_planning}
As we only have a single observation, our model has to learn the dynamics of the \ac{mdp} in order to predict the reward signal at time point $t > 1$. 
Since learning a transition model for an \ac{mdp} has a quadratic error bound in $\gamma$ \cite{NEURIPS2020_b5c01503}, we expect a quadratic error with the length of the horizon.
However, we identify that this challenge also grants us a unique improvement over previous actor critic methods.\\ 

By considering sequences of whole trajectories at once instead of single actions 
at time step $t$, 
we show that we get an unbiased approximation of the reward from the critic and can eliminate errors in the update step of the actor. First, we develop the objective function of the critic and 
show that it can be used to find optimal actions, then we develop the objective function of the actor and show that we prevent errors in the update step of conventional actor critic methods. 
We also expect this approach is robust to catastrophic forgetting induced by the distribution shift between expert demonstrations and reinforcement learning, 
as we will develop lower improvement bounds in Section \ref{sec:impro_bounds_avc}, which include the expert demonstrations.\\

\subsection{Critic}
\label{sec:AC_Critic}
To train the critic in actor critic methods, we can either use on-policy data as in section \ac{ppo} \ref{section:PPO}, which is not sample efficient, as only the data collected 
in the current episode is used for training, or we can use off-policy data as in section \ac{tqc} \ref{section:TQC}, where the critic has an overestimation bias. This 
leads to sub-optimal data usage and the reliance on an additional hyperparameters to compensate the overestimation. We find that in our setup with a single observation, we can use all data 
collected without introducing a bias on the critic. \\
Recall, that Thrun and Schwartz \cite{thrun1993issues} showed the overestimation bias comes from action noise $U(a)$ in the update step of the $Q$ value:
\begin{equation*}
    \max_{a}Q(s,a) = \max_{a}\ \mathbb{E}_U \left[Q(s,a) + U(a)\right]
\end{equation*}
\begin{equation}
    \leq \mathbb{E}_U \left[\max_a \{Q(s,a) + U(a)\}\right]
\end{equation}
We propose to use a whole sequence critic $C$, which predicts all immediate rewards given a current observation 
and the planned action sequence for the whole trajectory:
\begin{equation}
    \label{eq:emp_mean_c}
    C_{s_1, a_{1:T}}(t) = \mathbb{E}_{s_t \propto \tau_{s_1, a_{1:T}}}\left[r(a_t, s_t)\right],
\end{equation}
where $\tau_{s_1, a_{1:T}}$ indicates the sequence 
$$\tau_{s_1, a_{1:T}} = p(s_2, ..., s_T|s_1, a_{1:T}) = p(s_2|s_1, a_1) p(s_3|s_2,a_2)...$$
 until time step $T$. 
We define the objective function of the critic as:
\begin{equation}
    J_C = \mathbb{E}_{s_t \propto \tau_{s_1, a_{1:T}}}\left[(C_{s_1, a_{1:T}}(t) - r_t)^2\right].
\end{equation}
It is a standard result that the minimum of $J_C$ can be found at $C_{s_1, a_{1:T}}(t) = \mathbb{E}_{s_t \propto \tau_{s_1, a_{1:T}}}\left[r(s_t, a_t)\right]$. 
With limited sample size $\mathcal{D}$, this value is approximated with the empirical distribution $C_{s_1, a_{1:T}}(t) = \mathbb{E}_{s_t \propto \tau_{s_1, a_{1:T}} \in \mathcal{D}}\left[r(s_t, a_t)\right]$, 
which is an unbiased estimate.\\

\subsubsection{Optimality}
We want to show that the whole sequence critic can be used to find an optimal action sequence in a deterministic MDP with a deterministic transition function $T_{\text{det}}$.\\
Let $\pi^*$ be an optimal policy. We will use without a proof, that in a deterministic \ac{mdp}, $\pi^*$ can also be deterministic.\\
With $T_{\text{det}}$ the next state is determined by the current state and the action, so we get:
$$\pi^*(s_1) = a_1$$
$$s_2 = T_{\text{det}}(s_1, a_1)$$
$$a_2 = \pi(s_2) ..., $$
where we abused notation and wrote $s_{t+1} = T_{\text{det}}(s_t, a_t)$ for the deterministic transition function, where 
$\begin{cases}
    T_{\text{det}}(s', a_t, s_t) = 1\quad | s' = s_{t+1}\\
    T_{\text{det}}(s', a_t, s_t) = 0, \quad \text{else}.
\end{cases}$\\

We can now uniquely identify $a^*_{1:T} \propto \pi^*(s)$ as 
$$a^*_{1:T}(s) = \{a_1 = \pi^*(s) , a_2 = \pi^*(s_2|s_1,a_1), ... \},$$
where we only need $s$. \\
We can now write 
$$V_{\pi^*}(s) = V_{a^*_{1:T}(s)}(s)$$
$$=\mathbb{E}_{s_t \propto \tau_{s, a^*_{1:T}}}\left[\sum_{t=1}^T \gamma^t r(a_t, s_t)\right]$$
$$=\sum_{t=1}^T \gamma^t C_{s, a^*_{1:T}}(t),$$
where we used that the expectation commutes with summation. As $V_{\pi^*}(s)$ is the maximal value function, we conclude 
$$a^*_{1:T} = \max_{a_{1:T}}\gamma^t C_{s, a_{1:T}}(t),$$
since $C$ is an unbiased estimator of $r$. From this we conclude that finding the optimal action sequence with respect to the critic $C$ ensures the optimal 
cumulative reward of any policy. Following Section \ref{sec:prop_val_eq}, we find that for deterministic \ac{mdp}s, $C$ learns a proper value equivalent model $m'$ of the \ac{mdp} model $m$.

\subsection{Actor}
\label{sec:AC_actor}
In this section, we will assume a deterministic policy $\pi$. We propose the following argument can be generalized for stochastic actors but 
leave the formal derivation for future work and focus on the deterministic case of now.\\

All parametrized actor critic methods are technically speaking off-policy, as the critic value is bootstrapped with estimates sampled from former policies $\pi_{\mathrm{old}}$ during the training process. 
Once the actor is updated, the 
critic target on which the actor was updated has moved. Intuitively, when the policy is changed to choose a different action at time point $t$, it might also change 
the policy to choose different actions at 
all subsequent time points, which changes the value function. \\

Recall from Equation \ref{ac policy update}, the policy gradient in DDPG is calculated as:
\begin{equation}
    \nabla_{\phi} J(\phi, Q(\theta(\phi))) = \mathbb{E}_{\tau \sim p(\tau | \pi_{\phi})} \left[\nabla_{\phi} \log \pi(a_t|s_t;\phi) Q_{\theta, \pi_\phi}(a_t, s_t) \right].
\end{equation}
$Q_{\theta(\phi_i)}$ indicates the $Q$ function for policy $\pi_{\phi_{i}}$ with parameters 
$\theta(\phi_i)$. 
Generally, we do not have access to $\theta(\phi)$, but only an approximation given the limited data we have collected. Let us denote the actual 
$Q$ function we have as $Q(\theta(\phi) + \epsilon)$, where $\epsilon$ indicates an error between the optimal parameters and the actual parameters.
Let us now assume for simplicity, that $\phi$ and $\theta$ are one dimensional. A generalization to n dimensions is straight forward. We want 
to estimate the update error to $\pi$ from the objective $J$. While the proper derivative given optimal $\theta$ is given by $\nabla_{\phi} J(\phi, \theta(\phi))$,
the actor gradient $\widetilde{\nabla}_\phi J$ assumes $\theta$ is locally constant with respect to $\phi$ with $\widetilde{\nabla}_\phi J(\phi, \theta)$ 
We can express this in terms of the parameters $\theta(\phi) + \epsilon$ to 
find the actual update step $\widetilde{\nabla}_{\phi} J(\phi, \theta + \epsilon)$ computed in actor critic methods:
\begin{equation*}
    \widetilde{\nabla}_{\phi} J(\phi, \theta + \epsilon) = \widetilde{\nabla}_{\phi} J(\phi, \theta + \epsilon) - \widetilde{\nabla}_{\phi} J(\phi, \theta) + \widetilde{\nabla}_{\phi} J(\phi, \theta)
\end{equation*}
\begin{equation*}
    = \widetilde{\nabla}_{\phi} J(\phi, \theta) + \epsilon_{\widetilde{\nabla}_{\phi}},
\end{equation*}
with $\epsilon_{\widetilde{\nabla}_{\phi}} = \widetilde{\nabla}_{\phi} J(\phi, \theta + \epsilon) - \widetilde{\nabla}_{\phi} J(\phi, \theta)$.
From this, we can now derive the update error in the actor gradient step:
\begin{equation}
    \label{equation:general_update_error}
    \nabla_{\phi} J(\phi, \theta(\phi)) - \widetilde{\nabla}_\phi J(\phi, \theta + \epsilon) = \nabla_{\phi} J(\phi, \theta(\phi)) - \widetilde{\nabla}_\phi J(\phi, \theta + \epsilon) - \epsilon_{\widetilde{\nabla}_{\theta}}
\end{equation}
\begin{equation*}
    = \left(\nabla_{\phi} J(\phi, \theta(\phi) + \epsilon) - \widetilde{\nabla}_\phi J(\phi, \theta + \epsilon)\right) - \left( \epsilon_{\widetilde{\nabla}_{\theta}} + \epsilon_{\nabla_{\theta}} \right).
\end{equation*}
We will call 
\begin{equation}
    \label{eq:def_func_app_err}
     \epsilon_{\widetilde{\nabla}_{\theta}} + \epsilon_{\nabla_{\theta}}
\end{equation}
the \textbf{function approximator error}, as it is derived from the difference between 
the actual parameters $\theta(\phi) +\epsilon$ and the optimal parameters $\theta$, and \\
\begin{equation}
    \label{eq:def_di_sh_err}
    \nabla_{\phi} J(\phi, \theta(\phi) + \epsilon) - \widetilde{\nabla}_\phi J(\phi, \theta + \epsilon)
\end{equation}
the \textbf{distribution shift error}, as it is derived from the fact, that $\theta$ actually depends on $\phi$. \\
While this derivation assumes DDPG policy update, stochastic actor policy update methods like the minimization of a KL divergence as 
proposed in SAC will rely on similar gradients in their update step, see equation 13 of \cite{haarnoja2018soft}.

\subsubsection{Distribution Shift Error}
\label{dist_shift_error_section}
Following, we will show that the distribution shift error is zero using a whole sequence critic $C$ as defined in Equation \ref{eq:def_wsc}. \\ 

We established in the last section, that the distribution shift error is derived from the fact, that $\theta(\phi)$ is dependent on $\phi$. 
Recall, when we write $\theta(\phi)$ we mean $Q_{\theta}(a,s|\pi_{\phi})$:
\begin{equation*}
    Q_{\theta}(a,s|\pi_{\phi}) = \mathbb{E}_{(a_t \propto \pi_{\phi}(s_t), s_t \propto T(s_{t-1}, a_{t-1}, s_t))|a_1=a, s_1=s}\left[\sum_t \gamma^t r(a_t, s_t)\right]
\end{equation*}
where the dependence of $Q$ from $\phi$ lies in the distribution over which we take the expectation. Intuitively, $Q$ is dependent on $\phi$, because $\pi_{\phi}$ determines the 
future actions.\\ 

Recall, the definition of the whole sequence actor $C$ was: 
$$C_{\theta, s_1, a_{1:T}}(t) = \mathbb{E}_{s_t \propto \tau_{s_1, a_{1:T}}}\left[r(a_t, s_t)\right],$$
where we use parameters $\theta$ for the critic.\\
It can be seen, that $C_{\theta, s_1, a_{1:T}}$ has no dependency of $\pi$, as the action sequence is completely part of the argument. From this simple fact, we 
see the distribution shift error as defined in equation \ref{eq:def_di_sh_err} is zero. 

\subsubsection{Function Approximation Error}
\label{func_app_error}
Recall, the function approximator error $\epsilon_{\widetilde{\nabla}_{\theta}} + \epsilon_{\nabla_{\theta}}$ defined in Equation \ref{eq:def_func_app_err} was 
derived from the observation, that $\theta$ is underdetermined 
given limited data $\mathcal{D}^n$. With the whole sequence critic, $\epsilon_{\widetilde{\nabla}_{\theta}} = 0$, as we have no dependency from $C$ to $\pi$, 
but $\epsilon_{\nabla_{\theta}} \neq 0$, as $C$ is still a function approximator for $r$.\\

Intuitively, while we have the actual value $C_{s_1, a_{1:T}}(t), (s_1, a_{1:T}) \in \mathcal{D}^n$ of a given action sequence in our database $\mathcal{D}^n$, 
we don't know the value of an arbitrary action sequence starting from $s_1$: $C_{s_1, \tilde{a}_{1:T}}(t), (s_1 \in \mathcal{D}^n, \tilde{a}_{1:T} \notin \mathcal{D}^n)$. \\
To mitigate this error, we want to decouple the actor update from the critic by rephrasing the actor update as an \ac{il} problem. \\
Let's say we have a dataset $\mathcal{D}_E$ of expert demonstrations. Like in \ac{il}, we define the objective $J_{\phi}$ as:
\begin{equation}
    \label{eq:ac_obj_1}
    J_{\phi} = \mathbb{E}_{(o, a_{t}) \in \mathcal{D}_E, t \in \{1, ..., T\}}\sum_n \left(\pi_{\phi}(o)_{t, n} - a_{t, n}\right)^2,
\end{equation}
where we use a whole sequence actor 
$\pi(o):\mathcal{R}^m \rightarrow \mathcal{R}^{T \times N}$ with an $N$ dimensional action space and $T$ time steps per trajectory. The index $t$ denotes the $t'$-th action of the sequence and the index $n$ denotes the $n$th dimension of 
the action. For probabilistic policies, we can use the update rule as stated in \ref{prob_imitation_learning}.\\ 
The idea is, we don't push the actor towards a predicted goal, 
but only towards measured goals. This formulation allows us to use a good lower bound on the expected policy improvement.\\

\subsubsection{Policy Improvement}
\label{sec:impro_bounds_avc}
In the following, we will provide a lower bound for the policy improvement, which uses results from \ac{trpo} (Section \ref{sec:TRPO}). 
Assume we have a dataset $\mathcal{D}_{\text{E}}$ sampled from the expert policy $\pi_{\text{E}}$ with optimal value 
$$J(\pi_{\text{E}}) =\mathbb{E}_{\tau \propto \pi}\left[\sum_{t=0}^{\infty}  \gamma^t r(s_t) \right],$$
and a dataset $\tilde{\mathcal{D}}_{\text{E}}$ sampled from $\pi_{\text{E}}$ with $\mathcal{D}_{\text{E}} \subset \tilde{\mathcal{D}}_{\text{E}}$. 
We want to show, that a policy $\tilde{\pi}$ that is trained with \ac{il} on $\tilde{\mathcal{D}}_{\text{E}}$ has 
a higher expected reward then a policy $\pi$ trained on $\mathcal{D}_{\text{E}}$: $J_{\tilde{\pi}} \geq J_{\pi}$.
We define 
$$L_{\pi_{\text{E}}}(\pi) = J(\pi_{\text{E}}) + \sum_s \rho_{\pi_{\text{E}}}(s) \sum_a \pi(a|s) A_{\pi}(s,a),$$
where $\pi$ is the policy we optimize with respect to Equation \ref{eq:ac_obj_1} and $\rho_{\pi_{\text{E}}}(s)$ is known from 
the data distribution $\mathcal{D}_{\text{E}}$. We now get 
$$J({\pi}) \geq L_{\pi_{\text{E}}}({\pi}) - C D^{\max}_{\operatorname{KL}} (\pi,\pi_{\text{E}}).$$
Now if we optimize $\pi$ until convergence on the expert dataset, we get 
$$D^{\max}_{\operatorname{KL}}(\pi(s,a),\pi_{\text{E}}(s,a)) = 0 |(s,a) \in \mathcal{D}_{\text{E}}$$
and 
$$D^{\max}_{\operatorname{KL}}(\pi(s,a),\pi_{\text{E}}(s,a)) \geq 0 |(s,a) \notin \mathcal{D}_{\text{E}}.$$
It follows that 
$$D^{\max}_{\operatorname{KL}}(\tilde{\pi},\pi_{\text{E}}(s,a)) \leq D^{\max}_{\operatorname{KL}}(\pi(s,a),\pi_{\text{E}}(s,a)),$$
since strictly more action state pairs are in $\tilde{\mathcal{D}}$, the policy $\tilde{\pi}$ minimizes the KL divergence at more points.
On the seen dataset, $L_{\pi_{\text{E}}}({\pi}) = L_{\pi_{\text{E}}}({\tilde{\pi}})$, since there $\pi = \pi_{\text{E}}$, we finally get 
\begin{equation}
    J(\tilde{\pi}) \geq J({\pi}).
\end{equation}
A more rigorous proof, specifically considering the assumption $L_{\pi_{\text{E}}}({\pi}) = L_{\pi_{\text{E}}}({\tilde{\pi}})$ is left for future work.\\

This leaves us with the question of how to append the dataset $\mathcal{D}_{\text{E}}$ to improve our policy beyond what it can learn from the 
initial set of expert trajectories.\\

To do this, we propose to use 
a search based method using the critic in inference time. \\

Let $a = \pi(s)$ be the action sequence proposed by the actor and $C_{\theta, s, a_{1:T}}(t)$ be the predicted rewards by the critic. 
We can now find the gradient of the predicted cumulative reward $J(a) = \sum_{t=1}^{T} C_{\theta, s, a_{1:T}}(t)$ with respect to the actions to define new actions: 
$$a_{i+1} = a_i + \alpha_{\mathrm{inf}} \nabla_{a} J(a_i),$$ 
with inference time learning rate $\alpha_{\mathrm{inf}}$, as the predicted rewards are a differentiable function of the actions given the critic. This step is repeated k times, to find new actions. 
Using the critic "actively" in inference time is is the reason, why we call this method Active Critic.\\ 

The new trajectory $(r, s, a)$, found by using the critic, will 
be added to the set of trajectories for the critic $\mathcal{D}_c$ and to the set of expert trajectories $\mathcal{D}_{\text{E}}$, if it was a success with $r_T = 1$. \\

With this setup, our algorithm is truly on-policy, while being able to use all data collected from expert demonstrations and during train time. \\

The actor is model free, while the critic is weakly grounded model-based, as it learns to predict the reward dynamics of the \ac{mdp}. Tt is close to MuZero \cite{MUZero} in this sense, 
but our architecture enables search on continuous action space, which is not possible in MuZero. 

\section{Average Actions Problem}
\label{avr_action_problem}
In our lower bound for the policy improvement, we assume, we can find a policy that sets the KL divergence to the expert policy to zero. Our imitation loss is given by the $L_2$ distance to the 
observed data. It will converge to the mean of the trajectories in the dataset, given observation $s$. This will only set the KL divergence to 0, if the actions are chosen deterministically, 
thus the means of all trajectories with the same input observation is equal to those trajectories.\\

In our algorithm, most trajectories in the actor dataset are the result of the search for suitable actions given the current critic. We could find
multiple trajectories for the same observation, given that usually multiple action sequences will successfully solve a problem. It is not guaranteed, that the 
average trajectory will also solve the problem.\\ 

To illustrate this point, let's say our objective 
is to navigate from point A to point B around an obstacle C. Assume we found two successful trajectories 
$a_1$ and $a_2$ going left and right around C. Given our algorithm, both trajectories are added to the actor dataset. \\

The actor is now trained to minimize the $L_2$ distance, thus it will output the average of $a_1$ and $a_2$, which might be a straight line from A to B 
through C and thus not a successful trajectory. \\

To prevent this problem, we will disambiguate the actions $a_1$ and $a_2$. We propose an additional network we call the planner $P$. $P$ takes 
the action sequence $a$ and outputs a low dimensional encoding we call the plan $p_a$ of the sequence $a$: 
$P:\mathcal{R}^{T\times n} \rightarrow \mathcal{R}^{d_p},\ p(a) = p_a$, with action dimension $n$, sequence length $T$ and plan dimension $d_p$.\\ 
This encoding is then fed into the actor together 
with the observation $s$: $a = \pi(s, p_a)$. Now the actor can learn both trajectories disjunctly.\\ 

However we obviously don't have access to an action sequence during inference time to give to the planner $P$, 
that would solve the problem. Instead, we define the output of the planner for all action sequences that were initially provided as 
expert trajectories as the zero vector. We can do this, as long as the expert, from which the initial trajectories were sampled, is deterministic, or we make sure 
no two same observations are part of the initial expert trajectories. In inference time, we can then set the initial plan to the zero vector. An overview of the learning 
paradigm for the actor and planner is shown in Figure \ref{fig:actor_planner}.

\begin{figure}
    \centering
    \begin{tikzpicture}
        \node[draw, fill=blue!20, rounded corners] (obs_inpt) at (0,0) {$\begin{matrix}[o_1\end{matrix} ]$};
        \node[draw, fill=blue!20, rounded corners] (obs_rep) at (2,0) {$\left[\begin{matrix}o_1\\ \vdots\\ \times T \\ \vdots \\ o_1\end{matrix}\right]_t$};
        \node[draw, thick, circle] (embplus_1) at (4, 0) {$\oplus$};
        \node[draw, thick, circle, inner sep=0pt, label={[align=left]above:Positional\\ Encoding}] 
        (pe_1) at (4,1.5) {\tikz \draw[scale=0.1] plot[domain=0.0:6.28] (\x,{sin(\x r)});};
        \node[draw, fill=blue!20, rounded corners] (seq_enc_obs_rep) at (6.5,0) {$\left[\begin{matrix}\tilde{o}_1\\ \vdots \\ \tilde{o}_T\end{matrix}\right]_t$};


        \node[draw, fill=blue!20, rounded corners] (actions) at (0,5) {$\left[\begin{matrix}a_1^E\\ \vdots\\ a_T^E\end{matrix}\right]_t$};
        \node[draw, fill=black!20, rounded corners, minimum width=2cm, minimum height=1cm] (planner) at (3,5) {Planner};
        \node[draw, fill=blue!20, rounded corners] (plans) at (6.5,5) {$\begin{matrix}[p_1, ..., p_T\end{matrix} ]^T$};

        \node[draw, fill=black!20, rounded corners, minimum width=2cm, minimum height=1cm] (actor) at (6.5,3) {$\text{Actor}_{t\rightarrow \color{black!50!green!80} t+1}$};
        \node[draw, fill=blue!20, rounded corners, right = of actor] (prop_actions) {$\left[\begin{matrix}a_1\\ \vdots\\ a_T\end{matrix}\right]_t$};
        \node[draw, thick, rectangle , right = of prop_actions] (loss) {$|a_t - a^E_t|_{_2}^{^2}$};

        \draw[->, black!50, line width=1pt] (obs_inpt) to (obs_rep);
        \draw[->, black!50, line width=1pt] (obs_rep) to (embplus_1);
        \draw[->, black!50, line width=1pt] (pe_1) to (embplus_1);
        \draw[->, black!50, line width=1pt] (embplus_1) to (seq_enc_obs_rep);
        \draw[->, black!50, line width=1pt] (seq_enc_obs_rep) to (actor);

        \draw[->, black!50, line width=1pt] (actions) to (planner);
        \draw[->, black!50, line width=1pt] (planner) to (plans);
        \draw[->, black!50, line width=1pt] (plans) to (actor);

        \draw[->, black!50, line width=1pt] (actor) to (prop_actions);
        \draw[->, black!50, line width=1pt] (prop_actions) to (loss);

        \draw[->, black!50!green!50, line width=1pt, bend right = 30, dashed] (loss.west) to (prop_actions.east);
        \draw[->, black!50!green!50, line width=1pt, bend right = 30, dashed] (prop_actions.west) to (actor.east);
        \draw[->, black!50!green!50, line width=1pt, bend right = 30, dashed] (actor.north) to (plans.south);
        \draw[->, black!50!green!50, line width=1pt, bend right = 30, dashed] (plans.west) to (planner.east);



    \end{tikzpicture}
\caption{Overview of the learning step for the actor. The solid lines indicate the forward path, the dashed lines indicate backward gradients.}
\label{fig:actor_planner}

\end{figure}

\section{Inference Time Search}
\label{sec:inf_time_search}
In the derivation of the \ac{avc} algorithm, we have used the gradient signal from the critic on the actions. As our approach is completely parametrized, we can calculate 
the gradient with respect to the actor, the planner, the observation and the plan. Alet \etAl \cite{alet2021noether} propose a method that aims 
to learn useful conserved quantities to reconstruct a video sequence 
using conservation laws of the underlying physics. While a different problem setup, they also have a high dimensional derivative with respect to a 
generated output. The authors find, that 
calculating the gradient with respect to the generator and then regenerating the output performs better, than directly optimizing the output. They hypothesize this 
may be, because the generator has some knowledge about the action space. Following this insight, we found that updating the weights of the actor and planner 
with respect to the critic gradient works best. This update rule can be written as:
\begin{equation*}
    a^{i+1}_{\phi_{i+1}, \zeta_{i+1}} = \pi_{\phi_{i+1}}(s, p_{\zeta_{i+1}}(a^i))
\end{equation*}
\begin{equation*}
    \phi_{i+1} = \phi_i - \alpha_{\mathrm{inf}} \nabla_{\phi} (C_{\theta, s, a^{i}_{\phi_{i}, \zeta_{i}}}(T) - 1)^2
\end{equation*}
\begin{equation}
    \zeta_{i+1} = \zeta_i - \alpha_{\mathrm{inf}} \nabla_{\zeta} (C_{\theta, s, a^{i}_{\phi_{i}, \zeta_{i}}}(T) - 1)^2,
\end{equation}
where $\alpha_{\mathrm{inf}}$ is the inference time learning rate and $a^{i}_{\phi, \zeta}$ indicates the vector of $T$ actions parametrized by the 
actor $\pi_{\phi}$ and the planner $p_\zeta$. $p_{\zeta}(a^0) = 0$ for the first input to the actor network. \\

We do not apply the updated weights to the actor after the inference, but only use them in inference time for the given trajectory. While the update step to our 
actor has no distribution shift error as shown in Section \ref{dist_shift_error_section}, making the search highly efficient, we still have the function approximator error as 
shown in Section \ref{func_app_error}. Recall, the update to our actor is done by appending the expert dataset. An overview of the inference 
algorithm is shown in Figure \ref{fig:inference}

\begin{figure}
    \centering
    \begin{tikzpicture}
        \node[draw, fill=black!20, rounded corners, minimum width=2cm, minimum height=1cm] (planner) at (3,0) {$\text{Planner}_{t\rightarrow \color{black!50!green!80} t+1}$};
        \node[draw, fill=black!20, rounded corners, minimum width=2cm, minimum height=1cm] (critic) at (6,2) {Critic};
        \node[draw, fill=black!20, rounded corners, minimum width=2cm, minimum height=1cm] (actor) at (0,2) {$\text{Actor}_{t\rightarrow \color{black!50!green!80} t+1}$};
        
        \node[draw, fill=blue!20, rounded corners] (plans) at (0,0) {
            $\begin{cases}
                \begin{matrix}[p_1, ..., p_T\end{matrix} ]^\textbf{T}_{t>1}\\
                [0]_{t=0}
            \end{cases}$
            };
        \node[draw, fill=blue!20, rounded corners] (actions) at (3,2) {$\left[\begin{matrix}a_1\\ \vdots\\ a_T\end{matrix}\right]_t$};
        \node[draw, fill=blue!20, rounded corners] (exp_values) at (8.5,2) {$\left[\begin{matrix}c_1\\ \vdots\\ c_T\end{matrix}\right]_t$};
        \node[draw, thick, rectangle] (loss) at (11,2) {$|c^t_T - 1|_{_2}^{^2}$};

        \node[draw, fill=blue!20, rounded corners] (obs_1) at (0,5) {$\begin{matrix}[o_1,..., \times T, ..., o_1\end{matrix} ]^\textbf{T}$};
        \node[draw, fill=blue!20, rounded corners] (obs_2) at (6,5) {$\begin{matrix}[o_1,..., \times T, ..., o_1\end{matrix} ]^\textbf{T}$};

        \node[draw, thick, circle] (embplus_1) at (0,3.5) {$\oplus$};
        \node[draw, thick, circle] (embplus_2) at (6,3.5) {$\oplus$};

        \node[draw, thick, circle, inner sep=0pt,label={[align=left]left:Positional\\Encoding}] 
        (pe_1) at (-1.5,3.5) {\tikz \draw[scale=0.1] plot[domain=0.0:6.28] (\x,{sin(\x r)});};
        \node[draw, thick, circle, inner sep=0pt,label={[align=left]left:Positional\\Encoding}] 
        (pe_2) at (4.5,3.5) {\tikz \draw[scale=0.1] plot[domain=0.0:6.28] (\x,{sin(\x r)});};


        \draw[<-, black!50, line width=1pt] (plans)  to (planner);

        \draw[->, black!50, line width=1pt] (plans)  to (actor);

        \draw[->, black!50, line width=1pt] (actor) to (actions);
        \draw[->, black!50!green!50, line width=1pt, bend right = 30, dashed] (actions.west) to (actor.east);

        \draw[->, black!50, line width=1pt] (critic) to (exp_values);
        \draw[->, black!50!green!50, line width=1pt, bend right = 30, dashed] (exp_values.west) to (critic.east);

        \draw[->, black!50, line width=1pt] (actions) to (critic);
        \draw[->, black!50!green!50, line width=1pt, bend right = 30, dashed] (critic.west) to (actions.east);

        \draw[->, black!50, line width=1pt] (obs_1) to (embplus_1);
        \draw[->, black!50, line width=1pt] (embplus_1) to (actor);
        \draw[->, black!50, line width=1pt] (actions) to (planner);
        \draw[->, black!50!green!50, line width=1pt, bend right = 30, dashed] (actions.south) to (planner.north);


        \draw[->, black!50, line width=1pt] (pe_1) to (embplus_1);

        \draw[->, black!50, line width=1pt] (embplus_2) to (critic);
        \draw[->, black!50, line width=1pt] (obs_2) to (embplus_2);
        \draw[->, black!50, line width=1pt] (pe_2) to (embplus_2);

        \draw[->, black!50, line width=1pt] (exp_values) to (loss);
        \draw[->, black!50!green!50, line width=1pt, bend right = 30, dashed] (loss.west) to (exp_values.east);


    \end{tikzpicture}
\caption{Overview of the inference mechanism of \ac{avc}. The solid lines indicate the forward path, the dashed lines indicate backward gradients, which 
are used to update the actor from time step $t$ to \color{black!50!green!80} $t+1$ \color{black}.}
\label{fig:inference}

\end{figure}

\section{Algorithm}
\label{sec:algorithm}
In the following chapter, we will use $o_1$ to indicate the first observation. In our derivation, we assumed access to $s_1$. In our conducted experiments, $o_1 = s_1$ will hold.\\

We will first describe the inference of AVC, then the update step of the actor and finally the update step of the critic.

\subsection{Inference}
The inference of the actor critic algorithm using the gradient directly on the actions is depicted in algorithm \ref{AC_Inference}. 
The other optimization modes can be adapted straight forward. All models use a transformer encoder architecture. \\


\begin{algorithm}[H]
    Require $\mathcal{D}_{\text{critic}}$, $\mathcal{D}_{\text{expert}}$,  
    $C_{\theta}$, $\pi_{\phi}$, $\alpha_{\mathrm{inf}}$, $N$\\
    $p_a \gets 0^{T \times d_p}$ hfill \hfill Initialize plan to zero vector \\
    sample $o_1$ from MDP\\
    $\tilde{o}_{1, ..., T} \gets [o_1]^T \oplus [p_i]_{i=1}^T$ \hfill Add positional encoding $p_i$ to repeated observation \\
    $a^0 \gets \pi(\tilde{o}_{1, ..., T}, p_a)$\hfill Get initial actions $a^0$ according to policy $\pi$ \\
    \For {$i=1$ to $N$}{
        Update action $a^i$ using gradient descent:\\
        $a^{i} \gets a^{i-1} + \alpha_{\mathrm{inf}} \nabla_{a^{i-1}} (c(\tilde{o}, a^{i-1}) - 1)^2$
    }
    Execute actions sequence $a^N$\\
    Get reward $r_T$ based on final state\\
    Add experience to replay buffers:\\
    $\mathcal{D}_{\text{critic}} \gets \mathcal{D}_{\text{critic}} \cup (o_1, a^N_{1, ..., T}, r)$\\
    \If {$r = 1$}{
        $\mathcal{D}_{\text{expert}} \gets \mathcal{D}_{\text{expert}} \cup (o_1, a^N_{1, ..., T}, r)$
    }
    return $\mathcal{D}_{\text{expert}}$, $\mathcal{D}_{\text{critic}}$
    \caption{Active Critic Inference}
    \label{AC_Inference}

\end{algorithm}


Following the insights from the Section \ref{COD_AC}, we first repeat the input observation $o_1$ $T$ times and append a positional encoding $p_i$:
\begin{equation}
    \label{eq:seq_emb}
    \tilde{o}_{1, ..., T} = [o_1]^T \oplus [p_i]_{i=1}^T,
\end{equation}
with a transformer style sinusoidal encoding.
We then generate the plan encoding as a zero vector with $p_a = 0^{T \times d_p}$ with plan dimension $d_p$. The plan and the repeated observation is the input 
of the actor $\pi$:
\begin{equation}
    a^0 = a_{\pi, 1,...,T} = \pi(\tilde{o}_{1, ..., T}).
\end{equation}
The initial action sequence $a^0$ is then optimized given the critic. For example, if we optimize the actions directly, we get:
\begin{equation*}
    a^{i+1} = a^i + \alpha_{\mathrm{inf}}\nabla_{a} (C(o_1, a^i)(T) - 1)^2,
\end{equation*}
with the inference time search learning rate $\alpha_{\mathrm{inf}}$. Then, the trajectory is added to the respective data sets:
\begin{equation}
    \mathcal{D}_{\text{critic}} = \mathcal{D}_{\text{critic}} \cup (o_1, a^N, r_T)
\end{equation}
\begin{equation*}
    \mathcal{D}_{\text{expert}} = \mathcal{D}_{\text{expert}} \cup (o_1, a^N, r_T)\quad |\quad r_T = 1.
\end{equation*}
Note that with expert actions and expert dataset, we don't mean action sequences that were computed by 
a specific expert, for example a human, but rather generally an action sequence, which solves an \ac{mdp} given an observation $o$. The \ac{avc} learner might add 
additional successful trajectories to the expert dataset. \\

\subsection{Update Step Actor}
Now we want to derive the update step of the whole sequence actor as motivated in the Section \ref{sec:AC_actor}.
To prevent the average action problem defined in Section \ref{avr_action_problem}, we first encode the expert action sequence using our planner $p_{a_{\zeta}} = P_{\zeta}(a^{\text{E}}_{1, ..., T})$. 
We also build the sequence embedded observation sequence $\tilde{o}_{1, ..., T}$ as described in Equation \ref{eq:seq_emb}. We can now compute the actions as:
\begin{equation*}
    a_{\pi, 1,...,T} = \pi(\tilde{o}_{1, ..., T},p_{a_{\zeta}}).
\end{equation*}
From this we get the objective function of the actor:
\begin{equation}
    \label{actor_objective}
    J_{\phi, \zeta} = \mathbb{E}_{(o, a(t)) \in \mathcal{D}_E,\ t \in \{1, ..., T\}}\sum_n \left( \pi_{\phi}(\tilde{o}_{1, ..., T}, p_{a_{\zeta}})_{t, n} - a_{n}(t)\right)^2,
\end{equation}
with $n$ indicating the dimension of the action space. The respective gradient steps are:
\begin{equation*}
    \phi_{i+1} = \phi_i - \alpha_{\phi} \nabla_{\phi}J_{\phi, \zeta}
\end{equation*}
\begin{equation*}
    \zeta_{i+1} = \zeta_i - \alpha_{\zeta} \nabla_{\zeta}J_{\phi, \zeta}.
\end{equation*}
The pseudocode is depicted in algorithm \ref{Actor_Update_Alg}.
\begin{algorithm}
    \caption{Actor Update}
    \label{Actor_Update_Alg}
    Given expert demonstrations $\mathcal{D}_{\text{expert}}$, actor $\pi_{\phi}$, planner $P_{\zeta}$, 
    learning rate $\alpha_{\phi}$, learning rate $\alpha_{\zeta}$\\
     Sample $(a^{\text{E}}_{1, ..., T}, o_1)$ from $\mathcal{D}_{\text{expert}}$\\
    $\tilde{o}_{1, ..., T} \gets [o_1]^T \oplus [p_i]_{i=1}^T$ \hfill{repeat observations and append positional encoding} \\
    $p_a \gets $
    $\begin{cases}
        p_{\zeta}(a^{\text{E}}_{1, ..., T})\ |\ a^{\text{E}} \notin \text{original expert demonstrations}\\
        0^{T \times d_p}, \text{else}
    \end{cases}$ \hfill {generate plan $p_a$} \\
    $a^\pi_{1,...,T} \gets \pi_{\phi}(\tilde{o}_{1, ..., T}, p_a)$ \hfill{generate actions according to $\pi_{\phi}$} \\
    $J_{\pi} = \mathcal{L}_2(a^\pi_{1,...,T}, a^{\text{E}}_{1, ..., T})$ \hfill{compute $L_2$ loss from actions} \\
    $\phi \gets \phi_{\mathrm{old}} - \alpha_{\phi} \nabla_{\phi}J_{\pi}$\\
    $\zeta \gets \zeta{\mathrm{old}} - \alpha_{\zeta} \nabla_{\zeta}J_{\pi}$ \hfill {update $\phi$ and $\zeta$ according to gradients} \\
    return $\phi$, $\zeta$
\end{algorithm}

\subsection{Update Step Critic}
Using $\tilde{o}_{1, ..., T}$ as in \ref{eq:seq_emb}, the objective function of the whole sequence critic is defined as:
\begin{equation}
    J_{\theta} = \mathbb{E}_{(o, a_{1,...,T}, r_{1,...,T}) \in \mathcal{D}_{\text{critic}},\ t \in {1, ..., T}}\left[(C_{\theta, o, a_{1:T}}(t) - r(t))^2\right].
\end{equation}
The gradient step is:
\begin{equation*}
    \theta_{i+1} = \theta_i - \alpha_{\theta} \nabla_{\theta}J_{\theta}
\end{equation*}
with learning rate $\alpha_{\theta}$. The pseudocode is depicted in algorithm \ref{Critic_Update_Alg}.
\begin{algorithm}
    \caption{Critic Update}
    \label{Critic_Update_Alg}
    Given replay buffer $\mathcal{D}_{\text{critic}}$, critic $C_{\theta}$ and learning rate $\alpha_{\theta}$\\
    Sample $(a^{\text{E}}_{1, ..., T}, o_1, r)$ from $\mathcal{D}_{\text{critic}}$\\
    $\tilde{o}_{1, ..., T} \gets [o_1]^T \oplus [p_i]_{i=1}^T$ \hfill{repeat observations and append positional encoding} \\
    $c \gets C_{\theta}(\tilde{o}_{1, ..., T}, a^{\text{E}}_{1, ..., T})$\hfill{compute expected reward $c$} \\
    
    $J_{C} = \mathcal{L}_2(c, r)$ \hfill{compute $L_2$ loss from reward $r$} \\
    $\theta \gets \theta{\mathrm{old}} - \alpha_{\theta} \nabla_{\theta}J_{C}$\\
    return $\theta$
\end{algorithm}
\subsection{Active Critic Learner}
Putting everything together, we can now define the learning paradigm of the \ac{avc} algorithm, depicted in Algorithm \ref{AC_Learner}. The number of update 
steps in inference time $N$ can be used to weigh exploration against exploitation. A high number of update steps means the critic can find very different 
trajectories, to the one proposed by the actor. As these trajectories are further from the distribution on which the critic learned, the confidence of the 
critic score will be lower. Note, this hyperparameter has no effect on the performance of the actor, as we decoupled the critic from the actor.
\begin{algorithm}
    \caption{Active Critic Learner}
    \label{AC_Learner}
    Given expert demonstrations $\mathcal{D}_{\mathrm{expert}}$, sequence length $T$, learning rates $\alpha_{\mathrm{inf}}$, $\alpha_{\theta}$, 
    $\alpha_{\phi}$ and $\alpha_{\zeta}$, inference steps $N$ and update steps $N_{\mathrm{update}}$\\
     Initialize  $\mathcal{D}_{\mathrm{critic}} = \mathcal{D}_{\mathrm{expert}}$, 
    actor $\pi_{\phi}$, planner $P_{\zeta}$, critic $C_{\theta}$\\
    \While {\mathrm{not converged}}
    {
        \For {$s=1$ to $N_{\mathrm{samples}}$}
        {
           $\mathcal{D}_{\mathrm{expert}}$, $\mathcal{D}_{\mathrm{critic}}$ = Inference Step \hfill{see algorithm \ref{AC_Inference}}\\
        }
        \For {$i=1$ to $N_{\mathrm{update}}$}
        {
            $\phi_{new}$, $\zeta_{new}$ = Actor Update \hfill{see algorithm \ref{Actor_Update_Alg}}\\
            $\theta_{new}$ = Critic Update \hfill{see algorithm \ref{Critic_Update_Alg}}\\
        }
    }
\end{algorithm}

\section{Continued Observations}
\label{sec:relax_dense}
We now want to relax the constraint of only using one observation per trajectory to the deterministic \ac{mdp} setting. 
We refer to the typical \ac{mdp} setting as "continued" observations to differentiate it from our previous discussed setup with single (sparse) observations. 
Following this, we develop the addition to the \ac{avc} algorithm we call Continued Observation Active Critic (\ac{coavc}). 
We will first discuss how a straight forward adaption of \ac{avc} will have an unwanted failure mode. Then, we will develop an additional loss term to counter this problem.\\

So far we had one initial observation and repeated it $T$ times, where $T$ is the sequence length. We then used positional encoding to disambiguate the observations and generate an action sequence of 
length $T$ from it.\\ 

A straight forward adaptation of this algorithm to continued observations would be to use a different encoding of the observations. We propose to use a special token to indicate an 
observation from the future, that was not yet measured, $t_f$. We then build the current observation sequence at time $t$ by filling the $T - t$ unseen observations with $t_f$ and 
the planner to 
generate the current actions sequence $a^t_{1:T} = \pi(o_{1:t}, t_{f, t+1:T}, p(a))$. Similar to the loss function of the actor from the \ac{avc} algorithm, the new loss of the actor becomes: 
\begin{equation}
    \label{dense_actor_objective}
    J_{\phi, \zeta} = \mathbb{E}_{(o_{1:t}, a_{1:T}) \in \mathcal{D}_E,\ t \in \{1, ..., T\}}\left[\left( \pi_{\phi}(o_{1, ..., t}, t_{f, t+1:T}, p_{\zeta}(a))(t) - a(t)\right)^2\right],
\end{equation}
where we now have $T$ observation sequences per trajectory, as $o_{1:t}$ and $o_{1:t+1}$ are two different inputs.\\
With this construction, we can still use the whole sequence critic, 
which prediction is defined as $C_{o_{1:t}, t_{f, t+1:T}, a_{1:T}}(i) = \mathbb{E}_{\tau \propto a_{1:T}, o_{1:t}}\left[r(o_i, a_i)\right]$. The corresponding loss is
\begin{equation}
    \label{eq:dense_critic_loss}
    J_{\theta, \mathrm{MDP}} = \mathbb{E}_{(o_{1:t}, a_{1,...,T}, r_{1,...,T}) \in \mathcal{D}_{\text{critic}},\ t \in {1, ..., T}}\left[(C_{\theta, o_{1, ..., t}, t_{f, t+1:T},  a_{1,...,T}}(t) - r_t)^2\right].
\end{equation}

For simplicity, we will denote the prediction 
$C_a^t(i)$ and the actual expected reward $r(i)$, where $t$ is the current time step and $i$ denotes the time step of the predicted reward. If we mean the vector of all predicted rewards, we omit the 
parameter $i$ and write $C_a^t$.\\ 

Notice, that we have two time indices. The upper script index is the current time step of the trajectory seen by the critic. 
It determines which actions have already been taken and 
which observations are available for the critic. The parameter $(i)$ indicates to which time step of the trajectory the value belongs to. So for predictions, it indicates the prediction of the $i'$-th 
reward from the trajectory and for the reward $r(i)$ it indicates the actual reward from time step $i$.\\ 

In inference time, we choose the action sequence 
${a^t}^* = \max_{a^t} \sum_{i=t}^T C_{a^t}^t(i)$. The action at time step $t$ is then ${a^t}^*(t)$. We will show that with this adaptation, $\sum_{i=t}^T C_{{a^t}^*}^t(i)$ is not an unbiased estimate of 
$\sum_{i=t}^T r(i)$.\\

First, we note ${a^t}^* \neq {a^{t+1}}^*$ in general, as ${a^t}^* \propto C^t$ and ${a^{t+1}}^* \propto C^{t+1}$. $C^{t+1}$ sees an additional observation and can thus differ from $C^{t}$. 
From Equation \ref{eq:dense_critic_loss} we see, that the critic learns an estimate of the empirical distribution of action sequences that are present in the train dataset, so action sequences that 
were actually taken. As ${a^t}^*$ is generally not the action sequence enacted, but only provides the action at time step $t$, the critic does not learn an estimate $C_{{a^t}^*}^t$ of action sequence 
${a^t}^*$. \\

As $C_{{a^t}^*}^t$ is the maximum cumulative reward and we showed that $C_{{a^t}^*}^t$ is not learned, we conclude that $\sum_{i=t}^T C_{{a^t}^*}^t(i)$ 
will be an overestimation of the actual rewards $r \propto {a^t}^*$, where the action sequence ${a^t}^*$ is used for all actions of the trajectory. \\

To counter this, we propose the loss $J_{\mathrm{pred}, C}$ defined as:
\begin{equation}
    \label{proxy_loss_dense}
    J_{\mathrm{pred}, C} = \mathbb{E}_{o_{\textbf{1:t}} , o_{t+1}, a^t_{1:T} \propto \mathcal{D}_{\mathrm{train data}}}\left[ \sum_{i=1}^T(C_{{a^t}}^t(i) - C_{{a^t}}^{t+1}(i))^2 \right].
\end{equation}
Following, we will motivate how this loss can solve the problem. A rigorous proof 
is left for future work.\\

Intuitively, we demand that the critic expects the same rewards at time step $t$ and $t+1$, given action sequence $a^t$. When this loss is minimized, always taking the actions sequence ${a^t}^*$ maximising the estimated sum of 
rewards $C_{{a^t}^*}^t(i)$, 
the expected values $C^t$ will converge to the actual rewards received $r$. This is because minimizing the loss defined in \ref{proxy_loss_dense} ensures, that the expected 
values at time step $t+1$ must be equal or higher than the expected values at time step $t$, as the action sequence ${a^{t+1}}^* = {a^t}^*$ can be chosen with 
$C_{{{a^t}^*}}^t(i) = C_{{a^{t+1}}^*}^{t+1}(i)$. Now if $C_{{a^t}^*}^t(i) = C_{{a^t}^*}^{t+1}(i)$, the action sequence does not change in $t$: ${a^t}^* = {a^{t+1}}^*$, thus the action 
sequence ${a^t}^*$ is actually used for all actions of the trajectory and $C^t$ becomes an unbiased estimate, as ${a^t}^*$ is part of the train data set. \\

There are a few important observations. First, we did not show, that it is possible to learn an estimate such that $C_a^t = C_a^{t+1}$ for all action sequences $a$. 
In fact, this assumption relies on a deterministic 
environment. In a non-deterministic environment, $C_a^t(i)$ will be the mean of the posterior distribution over the rewards from the environment given the proposed action sequence $a$ 
and observations until time step $t$. \\

Moreover, it is not guaranteed, and in fact unlikely, that there is a unique optimal action sequence $a$. Even if $C^t = C^{t+1}$, ${a^t}^* \neq {a^{t+1}}^*$ in general. We can 
disambiguate action sequences with additional proxy losses. 
For now, we will ensure convergence by choosing the action sequence ${a^t}^*$ as a starting point for the optimization to get ${a^{t+1}}^*$, thus if $C^t = C^{t+1}$ we will find 
${a^t}^* = {a^{t+1}}^*$, because the gradient will be zero. 
The update 
steps are adapted straight forward from \ac{avc} using the new loss functions $J_{\theta} = J_{\mathrm{pred, C}} + J_{\theta, \mathrm{MDP}}$. 

\subsection{Inference Time Optimization}
A practical issue we came across is the inference time per step. This is a typical problem of model-based algorithms, as they need to optimize the trajectory given 
the current observations.\\ 

In our case, for each step the actor and planner are optimized given the critic predictions, thus the inference time 
per step is dependent on the number of optimization steps. If the number is low, learning is slow, as the proposed actions from the actor are barely changed. 
If the number is high, the computation time per trajectory is too long, as the total number of update steps is $n \times T$, with n updates per step and trajectory length 
T.\\ 

This problem did not occur with the \ac{avc} algorithm, as the action sequence is only optimized 
once per trajectory, limiting the additional computational cost.\\

Recall, we proposed in Section \ref{sec:inf_time_search} to use the gradient from the critic on the actor and recompute the 
actions, instead of optimizing the actions directly. We found a good solution is to keep the states of the actor and planner during inference of a trajectory, 
instead of resetting it after each step. After each inference, the actor is reset to the state before the inference optimization, as we only want to 
train it using \ac{il} from expert trajectories. With this adaptation, we can use a low number of optimization steps per time step and still change the trajectory meaningfully, 
as the update steps on the actor and planner accumulate during inference time.
     
